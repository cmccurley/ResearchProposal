\documentclass{article}[12 pt]
\usepackage{amssymb}
\usepackage{amsthm}
\usepackage{amsmath}
\usepackage{appendix}
\usepackage{array}
\usepackage{geometry}
\usepackage{enumitem}
\usepackage{graphicx}
\usepackage{subfig}
\usepackage{caption}
\usepackage{url}
\usepackage{float}
\usepackage{pdfpages}
\usepackage{shortvrb}
\usepackage{mathtools}
\usepackage{multirow}
\usepackage{hyperref}
\usepackage{commath}
\usepackage{bm}
\usepackage{dsfont}
\usepackage{blindtext}

\usepackage{listings}
\usepackage{color} %red, green, blue, yellow, cyan, magenta, black, white
\definecolor{mygreen}{RGB}{28,172,0} % color values Red, Green, Blue
\definecolor{mylilas}{RGB}{170,55,241}

\lstset{language=Matlab,%
    %basicstyle=\color{red},
    breaklines=true,%
    morekeywords={matlab2tikz},
    keywordstyle=\color{blue},%
    morekeywords=[2]{1}, keywordstyle=[2]{\color{black}},
    identifierstyle=\color{black},%
    stringstyle=\color{mylilas},
    commentstyle=\color{mygreen},%
    showstringspaces=false,%without this there will be a symbol in the places where there is a space
    numbers=left,%
    numberstyle={\tiny \color{black}},% size of the numbers
    numbersep=9pt, % this defines how far the numbers are from the text
    emph=[1]{for,end,break},emphstyle=[1]\color{red}, %some words to emphasise
    %emph=[2]{word1,word2}, emphstyle=[2]{style},    
}

\def\BibTeX{{\rm B\kern-.05em{\sc i\kern-.025em b}\kern-.08em
		T\kern-.1667em\lower.7ex\hbox{E}\kern-.125emX}}

\geometry{margin=1 in}

\newcommand{\smallvskip}{\vspace{5 pt}}
\newcommand{\medvskip}{\vspace{30 pt}}
\newcommand{\bigvskip}{\vspace{100 pt}}
\newcommand{\tR}{\mathtt{R}}

\graphicspath{{"C:/Users/Conma/Documents/ResearchProposal/Paper/Images/"}}

\begin{document}


\title{Literature Review}
\date{}
\maketitle


%===================================================
%==================== Background ===================
%===================================================
\section*{Manifold Learning}
    Classic manifolds, neural nets, graphs

\section*{Hierarchical Mixture of Experts (HME)}
    and related context-dependent methods

\section*{Representation Learning}
    Neural networks/ autoencoders

\section*{Classifier Fusion/ Selection}

\section*{Lifelong Learning (Growing and Merging)}

\section*{Time Series Adaptation}
    Merging/ Stitching (How  to merge, when to merge), dissimilarity metrics, graph adaptation?

\section*{Background Modeling?}



%===================================================
%=================== References=====================
%===================================================
\section*{References}
\bibliography{proposalBib}


% \begin{center}
% 	\begin{figure}[h!]
% 		\centering
% 		\includegraphics[width=0.65\textwidth]{"transDiagram"}
% 		\caption{State transition diagram modeled by $P$.}
% 		\label{fig:transDiagram}
% 	\end{figure}
% \end{center}



% \begin{center}
%     \begin{figure}[H]
%         \centering
%         \includegraphics[width=0.65\textwidth]{"HW04Q13"}
%         \caption{State probability histograms for 1000 simulations after a burn-in time of $n=25$.  It can be observed that each distribution is centered around the expected equilibrium value of 0.33.}
%         \label{fig:stateHists}
%     \end{figure}
% \end{center}



% \begin{center}
% 	\begin{figure}[h!]
% 		\centering
% 		\includegraphics[width=0.65\textwidth]{"hw02_q2_partb"}
% 		\caption{Conditional mean for X given Y (red) and quadratic estimator for X given Y (blue).}
% 		\label{fig:estimators}
% 	\end{figure}
% \end{center}

%  \begin{figure}[H]
%    \centering
%    \subfloat[][]{\includegraphics[width=.4\textwidth]{"hw02_q7_hist_point6"}}\quad
%    \subfloat[][]{\includegraphics[width=.4\textwidth]{"hw02_q7_hist_1"}}\\
%    \subfloat[][]{\includegraphics[width=.4\textwidth]{"hw02_q7_hist_2"}}\quad
%    \subfloat[][]{\includegraphics[width=.4\textwidth]{"hw02_q7_hist_5"}}
%    \caption{Histograms of scaled mean estimates for learning parameters a) $g=0.6$, b) $g=1$, c) $g=2$, and d) $g=5$.}
%    \label{fig:q7Hists}
% \end{figure}



\end{document}
