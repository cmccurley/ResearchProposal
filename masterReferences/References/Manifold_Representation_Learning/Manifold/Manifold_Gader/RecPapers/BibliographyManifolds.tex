\documentclass[journal]{IEEEtran}
\bibliographystyle{IEEEtran}%ieeetr}
%\bibliographystyle{abbrv}

\usepackage{graphics,graphicx, subfig, amsmath,url, listings, color}
%\graphicspath{{figs/}}

\usepackage{tikz}
\newcommand{\udash}[1]{%
    \tikz[baseline=(todotted.base)]{
        \node[inner sep=1pt,outer sep=0pt] (todotted) {#1};
        \draw[dashed] (todotted.south west) -- (todotted.south east);
    }%
}%

\begin{document}

\section{\textbf{Manifold Learning Seminar Bibliography}}




The first two papers are the original Local Linear Embedding and a follow-up paper.  \cite{OrigLLE, LLE} They should be presented together.

The second two papers are the original ISOMAP and a follow-up paper.  \cite{OrigISO, IsoMultiMan} They should be presented together as well.

The next two papers look at a less talked about algorithms, called Sammon's algorithm.  One is the original and the other is a later paper.  They should be presented together \cite{SammonOrig, SammonBregDiv}
The next two papers focus on Dimensionality Estimation, which is an important but unsolved problem. 

The next two papers consider the important problem of estimating dimensionality \cite{TensorVoteDimMan, EntropicGraphsDimMan}.  They do not need to be presented together

The next two papers considered behavior on Out Of Sample patterns \cite{OOSManClass, OOSManBengioEtAl}.  The do not need to be presented together.

The final three papers look at Spectral Clustering, Persistence Landscapes,  and Manifold Learning for Hyperspectral Image Analysis \cite{SpecClusMan, BubenikPersistent, ManFeatHSIReview, ManGraphSemiHSIClass}.  They do not need to be presented together.

\bibliography{Manifold}
\end{document}