\documentclass[]{article}

\usepackage[margin=1in]{geometry}
\usepackage[round]{natbib}
\usepackage{indentfirst}
\usepackage[hidelinks,pdfnewwindow=true]{hyperref}
\usepackage[dvipsnames]{xcolor}

\usepackage{amsmath,amsfonts,amssymb}
\usepackage{graphicx}
\usepackage{appendix}
\usepackage{array}
\usepackage{caption}
\usepackage{url}
\usepackage{float}
\usepackage{pdfpages}
\usepackage{shortvrb}
\usepackage{mathtools}
\usepackage{multirow}
\usepackage{commath}
\usepackage{bm}

\DeclareMathOperator*{\argmax}{arg\,max}
\DeclareMathOperator*{\argmin}{arg\,min}



%\graphicspath{{"C:/Users/Conma/Documents/2019_SPIE/Paper/Images/DSRF/"}{"C:/Users/Conma/Documents/2019_SPIE/Paper/Images/"}{"C:/Users/Conma/Documents/2019_SPIE/Paper/Images/png_figures_squished/"}}


%the following allows 5 deep section headings (can be useful for dividing things up)
%section
%  subsection
%    subsubsection
%      paragraph
%        subparagraph
\setcounter{secnumdepth}{5}
\setcounter{tocdepth}{5}

%defines list object
%inputs are
%[1] bibtex reference label
%[2] Paper title
%[3] pdf file name (folder is hard coded as ../References)
%[4] abstract or any text you want to add
\newcommand{\paperentry}[4]{
            \hangindent=1cm
            \cite{#1} - \href{run:../References/#3}{\textcolor{ForestGreen}{\textit{#2}}}
            
            \noindent            
            \begin{minipage}[t]{0.1\linewidth}\hfill\end{minipage}
            \begin{minipage}[t]{0.8\linewidth}\textcolor{NavyBlue}{{\textit{Summary:}}}#4\end{minipage}
            \vspace{.25cm}
          }

%opening
\title{List of References}

\author{Connor H. McCurley}

\date{}

\begin{document}

\maketitle

\tableofcontents

\newpage

      
%%%%%%%%%%%%%%%%%%%%%%%%%%% Manifold Learning %%%%%%%%%%%%%%%%%%%%%%%%%%%%%%%%%%%%%%%%%%%%
\section{Manifold/ Representation Learning}

	\subsection{Classic Methods}

	\paperentry{VanDerMaaten2009DRReview}
	{Dimensionality Reduction: A Comparative Review}
	{Manifold_Representation_Learning/Reviews/VanDerMaaten2009DRReview.pdf}
	{}
	
	\paperentry{Jindal2017ReviewDRTechniques}
	{A Review on Dimensionality Reduction Techniques}
	{Manifold_Representation_Learning/Reviews/Jindal2017ReviewDRTechniques.pdf}
	{}

	\paperentry{Bengio2014RepLearningReview}
	{Unsupervised Feature Learning and Deep Learning: {A} Review and New Perspectives}
	{Manifold_Representation_Learning/Reviews/Bengio2014RepLearningReview.pdf}
	{}
	
	\paperentry{Tenenbaum2000Isomap}
	{A Global Geometric Framework for Nonlinear Dimensionality Reduction}
	{Manifold_Representation_Learning/Manifold/Tenenbaum2000Isomap.pdf}
	{}
	
	\paperentry{Roweis2000LLE}
	{Nonlinear Dimensionality Reduction by Locally Linear Embedding}
	{Manifold_Representation_Learning/Manifold/Roweis2000LLE.pdf}
	{}
	
	\paperentry{Saul2001LLEIntro}
	{An introduction to locally linear embedding}
	{Manifold_Representation_Learning/Manifold/Saul2001LLEIntro.pdf}
	{}
	
	\paperentry{Belkin2003LaplacianEigenmaps}
	{Laplacian Eigenmaps for Dimensionality Reduction and Data Representation}
	{Manifold_Representation_Learning/Manifold/Belkin2003LaplacianEigenmaps.pdf}
	{}
	
	
	\paperentry{Bishop1998GTM}
	{GTM: The Generative Topographic Mapping}
	{Manifold_Representation_Learning/Manifold/Bishop1998GTM.pdf}
	{}
	
	\paperentry{Delaporte2008DiffusionMaps}
	{An introduction to diffusion maps}
	{Manifold_Representation_Learning/Manifold/Delaporte2008DiffusionMaps}
	{}
	
	\paperentry{Theodoris2008PCA}
	{The Karhunen-Loeve Transform}
	{}
	{}
	
	\paperentry{Theodoris2008KPCA}
	{Kernel PCA}
	{}
	{}
	
	\paperentry{Tipping1999PPCA}
	{Probabilistic Principal Component Analysis}
	{Manifold_Representation_Learning/Manifold/Tipping1999PPCA.pdf}
	{}
	
	\paperentry{Lawrence2003GPLVM}
	{Gaussian Process Latent Variable Models for Visualisation of High Dimensional Data}
	{Manifold_Representation_Learning/Manifold/Lawrence2003GPLVM.pdf}
	{}
	
	\paperentry{Lawrence2005PPCAGPLVModels}
	{Probabilistic Non-linear Principal Component Analysis with Gaussian Process Latent Variable Models}
	{Manifold_Representation_Learning/Manifold/Lawrence2005PPCAGPLVModels.pdf}
	{}
	
	\paperentry{Gorban2007ElasticMaps}
	{Elastic Maps and Nets for Approximating Principal Manifolds and Their Application to Microarray Data Visualization}
	{Manifold_Representation_Learning/Manifold/Gorban2007ElasticMaps.pdf}
	{}
	
	\paperentry{Lee2015MultipleManifolds}
	{Learning Representations from Multiple Manifolds}
	{Manifold_Representation_Learning/Manifold/Lee2015MultipleManifolds.pdf}
	{}
	
	\paperentry{Kokiopoulou2007OrthoNeighborhoodPreservingProjections}
	{Orthogonal Neighborhood Preserving Projections: A Projection-Based Dimensionality Reduction Technique}
	{Manifold_Representation_Learning/Manifold/Kokiopoulou2007OrthoNeighborhoodPreservingProjections.pdf}
	{}
	
	\paperentry{Talmon2015ManifoldLearningInDynamicalSystems}
	{Manifold Learning for Latent Variable Inference in Dynamical Systems}
	{Manifold_Representation_Learning/Manifold/Talmon2015ManifoldLearningInDynamicalSystems.pdf}
	{}
	
	\paperentry{Nickel2017PoincareEmbeddings}
	{Poincar\'{e} Embeddings for Learning Hierarchical Representations}
	{Manifold_Representation_Learning/Manifold/Nickel2017PoincareEmbeddings.pdf}
	{}
	
	%%%%%%%%%%%%%%%%%%%%%%%%%%%%%% Supervised %%%%%%%%%%%%%%%%%%%%%%%%%%%%%%%%%%%%%%%%%%%%%%
	\subsection{Supervised and Semi-Supervised}
	
	
	\paperentry{Belkin2004SemiSupLearningRiemannianManifolds}
	{Semi-Supervised Learning on Riemannian Manifolds}
	{Manifold_Representation_Learning/Supervised/Belkin2004SemiSupLearningRiemannianManifolds.pdf}
	{}
	
	\paperentry{Geng2005SupNonlinearDimRed}
	{Supervised nonlinear dimensionality reduction for visualization and classification}
	{Manifold_Representation_Learning/Supervised/Geng2005SupNonlinearDimRed.pdf}
	{}
	
	\paperentry{Sugiyama2006FDASupDimRed}
	{Local Fisher Discriminant Analysis for Supervised Dimensionality Reduction}
	{Manifold_Representation_Learning/Supervised/Sugiyama2006FDASupDimRed.pdf}
	{}
	
	\paperentry{Rish2008SupDimRedGLM}
	{Closed-form supervised dimensionality reduction with generalized linear models}
	{Manifold_Representation_Learning/Supervised/Rish2008SupDimRedGLM.pdf}
	{}
	
	\paperentry{Zhang2008SpectralSemiSupManifoldLearning}
	{Spectral methods for semi-supervised manifold learning}
	{Manifold_Representation_Learning/Supervised/Zhang2008SpectralSemiSupManifoldLearning.pdf}
	{}
	
	\paperentry{Li2009SupManifoldLearning}
	{A supervised manifold learning method}
	{Manifold_Representation_Learning/Supervised/Li2009SupManifoldLearning.pdf}
	{}
	
	\paperentry{Gao2011SupGPLVMDimRed}
	{Supervised Gaussian Process Latent Variable Model for Dimensionality Reduction}
	{Manifold_Representation_Learning/Supervised/Gao2011SupGPLVMDimRed.pdf}
	{}
	
	
	\paperentry{Raducanu2012SupervisedNonlinearDimReduction}
	{A supervised non-linear dimensionality reduction approach for manifold learning}
	{Manifold_Representation_Learning/Supervised/Raducanu2012SupervisedNonlinearDimReduction.pdf}
	{}
	
	\paperentry{Gonen2013BayesianSupDimRed}
	{Bayesian Supervised Dimensionality Reduction}
	{Manifold_Representation_Learning/Supervised/Gonen2013BayesianSupDimRed.pdf}
	{}
	
	\paperentry{Liu2014HybridManifoldEmbedding}
	{Hybrid Manifold Embedding}
	{Manifold_Representation_Learning/Supervised/Liu2014HybridManifoldEmbedding.pdf}
	{}
	
	\paperentry{Bouzas2015GraphEmbeddedMutualInformationSupDimRed}
	{Graph Embedded Nonparametric Mutual Information for Supervised Dimensionality Reduction}
	{Manifold_Representation_Learning/Supervised/Bouzas2015GraphEmbeddedMutualInformationSupDimRed.pdf}
	{}
	
	\paperentry{Vural2016OutOfSampleSupManifoldLearning}
	{Out-of-Sample Generalizations for Supervised Manifold Learning for Classification}
	{Manifold_Representation_Learning/Supervised/Vural2016OutOfSampleSupManifoldLearning.pdf}
	{}
	
	
	\paperentry{Vepakomma2016SupDimRedDistanceCorrelationMax}
	{Supervised Dimensionality Reduction via Distance Correlation Maximization}
	{Manifold_Representation_Learning/Supervised/Vepakomma2016SupDimRedDistanceCorrelationMax.pdf}
	{}
	
	\paperentry{Xu2017ActiveManifoldLearning}
	{Active manifold learning via a unified framework for manifold landmarking}
	{Manifold_Representation_Learning/Supervised/Xu2017ActiveManifoldLearning.pdf}
	{}
	
	\paperentry{Vural2018StudySupervisedManifoldLearning}
	{A study of the classification of low-dimensional data with supervised manifold learning}
	{Manifold_Representation_Learning/Supervised/Vural2018StudySupervisedManifoldLearning.pdf}
	{}
	
	
	\paperentry{Kang2018ManifoldRegularizedPCA}
	{Robust Graph Learning from Noisy Data}
	{Manifold_Representation_Learning/Supervised/Kang2018ManifoldRegularizedPCA.pdf}
	{}
	

	\paperentry{Chen2018RobustSemiSupManifoldLearning}
	{Robust Semi-Supervised Manifold Learning Algorithm for Classification}
	{Manifold_Representation_Learning/Supervised/Chen2018RobustSemiSupManifoldLearning.pdf}
	{}
	
	\paperentry{Chao2019RecentAdvancesSupervisedDimRed}
	{Recent Advances in Supervised Dimension Reduction: A Survey}
	{Manifold_Representation_Learning/Supervised/Chao2019RecentAdvancesSupervisedDimRed.pdf}
	{}
	
	
	%%%%%%%%%%%%%%%%%%%%%%%%%%%%%%%%%% Manifold Alignment %%%%%%%%%%%%%%%%%%%%%%%%%%%%%%%%%%
	\subsection{Manifold Alignment/Matching}
	
		\paperentry{graphMatching}
		{arg2}
		{arg3}
		{arg4}
	
		\paperentry{Navaratnam2007JointManifoldSemiSupRegression}
		{The Joint Manifold Model for Semi-supervised Multi-valued Regression}
		{Manifold_Representation_Learning/Alignment/Navaratnam2007JointManifoldSemiSupRegression.pdf}
		{arg4}
	
		\paperentry{Wang2011ManifoldAlignment}
		{Manifold Alignment}
		{Manifold_Representation_Learning/Alignment/Wang2011ManifoldAlignment.pdf}
		{} \\ \textbf{\textit{Manifold alignment} is a simultaneous solution to the problem of alignment and a framework for discovering a unifying representation of multiple datasets.  The goal is  to ultimately map disparate datasets to a joint latent space, while preserving the qualities of each dataset and also highlighting the similarities between the datasets. The fundamental idea of manifold alignment is that all datasets included in the fusion lie on the same manifold.} \\ \\
			
		The problem of alignment is to identify a transformation of one dataset that ``matches it up" with a transformation of another dataset.  That is, given two datasets $X$ and $Y$, whose instances lie on the same manifold, $Z$, but who may be represented by different features, the problem of alignment is to find two functions $f$ and $g$, such that $f(x_i)$ is close to $g(y_j)$ in terms of Euclidean distance if $x_i$ and $y_j$ are close with respect to geodesic distance along $Z$.  Here, $X$ is an $n \times p$ matrix containing $n$ data instances in $p$-dimensional space, $Y$ is an n$m\times q$ matrix containing $m$ data instances in $q$-dimensional space, $f:\mathbb{R}^{p}\rightarrow\mathbb{R}^{k}$, and $g:\mathbb{R}^{q}\rightarrow\mathbb{R}^{k}$ for some $k$ called the latent dimensionality.  The instances $x_i$ and $y_j$ are in exact correspondence if and only if $f(x_i)=g(y_j)$.  On the other hand, prior correspondence information includes any information about the similarity of the instances in $X$ and $Y$, not just exact correspondence information. The union of the range of $f$ and $g$ is the joint latent space, and the concatenation of the new coordinates \[ \left( \begin{array}{cc}
		f(X) \\
		g(Y)  \end{array} \right)\] 
		is the unified representation of $X$ and $Y$, where $f(X)$ is an $n\times k$ matrix containing the result $f$ applied to each row of $X$, and $g(Y)$ and $m \times k$ matrix containing the result of $g$ applied to each row of $Y$.  $f(X)$ and $g(Y)$ are the new coordinates of $X$ and $Y$ in the joint latent space.
		\\ \\	
		This paper captures this idea mathematically by concatenating the graph Laplacians of each dataset, thus forming a joint Laplacian.  A within-dataset similarity function gives the edge weights of this joint Laplacian between the instances within each dataset, and correspondence information fills in the edge weights between the instances in separate datasets.  The manifold alignment algorithm then embeds this joint Laplacian in a new latent space.  \textbf{The primary challenges of manifold alignment are identifying whether the datasets are actually sampled from a single underlying manifold, defining the similarity function that captures the appropriate structures of the datasets, inferring any reliable correspondence information, and finding the true dimensionality of this underlying manifold.} Experiments were conducted on protein alignment, parallel corpora and aligning topic models. \\ \\Manifold alignment is essentially ta graph-based algorithm, but there is also a vast literature on graph-based methods for alignment that are unrelated to manifold learning, typically called \textit{graph matching} or \textit{graph isomorphism}. Researchers study the general problem of alignment under the names of \textit{information fusion} or \textit{data fusion}. \\
		
		
		\paperentry{Davenport2010JointManifoldsDataFusion}
		{Joint Manifolds for Data Fusion}
		{Manifold_Representation_Learning/Alignment/Davenport2010JointManifoldsDataFusion.pdf}
		{ In sensor networks, multiple observations of the same event acquired simultaneously from interdependent signals often share a common parameterization.  As examples, a camera network might observe a single event from a variety of angles where the underlying event is described by a set of common global parameters (i.e. location, orientation of the object of interest).  Similarly, when sensing a single event using multiple modalities, the underlying phenomenon may again be described by a single parameterization (which spans all modalities).  The authors contend that a joint manifold can be obtained which encompasses all the component manifolds sharing the same parameterization.
		\\ \\
		The authors compare properties such as geodesic distances, curvature, branch separation, and condition number of the joint manifold in relation to their component manifolds.  They observe that the joint manifold leads to improved performance and noise tolerance for a variety of signal processing algorithms (including ATR). The authors also show how the joint manifold structure can be exploited via a data fusion algorithm based on \textit{random projections}.
		\\ \\
		There is an argument that \textit{linear} dimensionality reduction should be performed to allow for out-of-sample embedding of the individual modalities, thus allowing for fusion based on compressive sensing and random projections.
		\\ \\
		The problem of manifold-based classification is, given a set of manifolds, to determine which generated a particular sample.  This problem has been explored for application to ATR. The authors describe three distances to be used for generalized maximum likelihood.  These distances, which define separation in metric spaces are: \textit{minimum separation}, \textit{maximum separation}, and \textit{Hausdorff distance}.
		\\ \\
		After reading this paper, it is not quite clear how this method would be implemented. 
		} \\
		
		\paperentry{Wang2010MultiscaleManAlignment}
		{Multiscale Manifold Alignment}
		{Manifold_Representation_Learning/Alignment/Wang2010mMultiscaleManAlignment.pdf}
		{}\\
		One limitation to existing domain adaptation problems is the assumption that source and target domains are defined by the same features and the differences come from their distributions.  To address this problem, manifold alignment was proposed.  Manifold alignment builds mappings between disparate data sets by aligning their underlying manifolds, thus providing a geometric framework for knowledge transfer across data sets.
		\\ \\
		Existing manifold alignment approaches can be classified into two types. In two-step alignment, the first step maps the data to low-dimensional spaces, then a subsequent step eliminates some components from one set to the other so alignment can be achieved.  One-step alignment combines the embedding projection and alignment steps into one.  Two approaches for one-step alignment are \textit{semi-supervised alignment} and \textit{manifold projections}. Semi-supervised alignment learns \textit{instance-level alignment} by constructing non-linear embeddings, while manifold projections learn \textit{feature-level alignment} by constructing linear embedding functions, thus allowing for out-of-sample embeddings.
		\\ \\
		However, many real-world data sets exhibit non-trivial regularities at multiple levels, which correspond to their underlying intrinsic structure.  With this in mind, the authors investigated the previously un-studied notion of multiscale manifold alignment by using multi-resolution wavelet analysis (which discovers multiscale intrinsic structure in a matrix).  Multiscale manifold alignment automatically generates alignment results at different scales by discovering the shared intrinsic multilevel structures exhibited by the datasets.  This provides a solution to the key problem in manifold alignment where users must define the latent dimensionality.  \\
		
		\paperentry{Wang2011HeteroDomainAdaptationManAlignment}
		{Heterogeneous Domain Adaptation Using Manifold Alignment}
		{Manifold_Representation_Learning/Alignment/Wang2011HeteroDomainAdaptationManAlignment.pdf}
		{} \\
		This paper proposed a manifold alignment based approach for heterogeneous domain adaptation.  A key aspect of this approach is to construct mappings to link different feature spaces in order to transfer knowledge across domains.  The new approach can reuse labeled data from multiple source domains in a target domain even in the case that the input domains do not share any common features or instances.  This paper extended existing manifold alignment approaches by making use of labels, rather than correspondences, to align the manifolds.  This was a significant extension to the scope of manifold alignment, as correspondence relationships required by previous methods is often difficult to obtain. \\ \\
		One limitation of domain adaptation is that most existing approaches assume that the sources and target domains are defined by the same features, and the differences between them primarily stem from their distributions.  However, this assumption is not valid in most cross-domain applications such as multimodal datasets.
		\\ \\
		A key difficulty in applying manifold alignment to domain adaptation is that alignment methods require specifying a small amount of cross-domain correspondence relationships in order to learn mapping functions.  However, this correspondence information is often difficult to obtain.
		\\ \\
		The contributions of this paper are two-fold.  From the perspective of domain adaptation, the new approach addresses the problem of transfer even when the source and target domains do not share any common features or instances.  From the perspective of manifold alignment, the new  approach uses labels rather than correspondences to learn alignment.
		\\ \\
		Here the authors treat each input dataset as a manifold.  The goal is to construct $K$ mapping functions to project the input domains to a new latent space while preserving the topology of each domain, matching instances with the same labels and separating instances with different labels.  Each domain is represented by its graph Laplacian (which is built by instance similarity, label similarity, and label dissimilarity).  A cost function is constructed such that trade-offs between the three objectives can be met.  This function is optimized to learn the set of mapping functions. By representing each domain by it's graph Laplacian, the issue of varying feature sizes is nullified. \\
		
		\paperentry{Choo2012HeterogeneousDataFusionGraphs}
		{Heterogeneous Data Fusion via Space Alignment Using Nonmetric Multidimensional Scaling}
		{Manifold_Representation_Learning/Alignment/Choo2012HeterogeneousDataFusionGraphs.pdf}
		{}\\
		Heterogeneous data sets are typically represented in feature spaces, making it difficult to analyze relationships spanning different data sets even when they are semantically related. Data fusion via space alignment can remedy this task by integrating multiple data sets lying in different spaces into one common space. Given a set of reference correspondence data that share the same semantic meaning across different spaces, space alignment attempts to place the corresponding reference data as close together as possible, and accordingly, the entire data are aligned in a common space. Space alignment involves optimizing two	potentially conflicting criteria: minimum deformation of the
		original relationships and maximum alignment between the different spaces. To solve this problem, we provide a novel	graph embedding framework for space alignment, which converts each data set into a graph and assigns zero distance between reference correspondence pairs resulting in a single graph. We propose a graph embedding method for fusion based on nonmetric multidimensional scaling (MDS). 
		\\ \\
		The goal of data fusion via space alignment is, given data sets and their reference correspondence pairs (some pairs of data items that share the same semantic meaning between different feature spaces), to provide the new representation of the data in a common space, which enables distance-based comparison across different data sets.   The idea is to attain the best alignment between different spaces while allowing a minimum deformation within each space.  The authors introduce a novel graph embedding based off of nonmetric MDS, which unlike metric MDS which aims to preserve given pairwise similarities, preserves only the rank ordering of the dissimilarities.
		\\ \\
		Graph embedding takes as input a graph in which the vertices are data items and the edge weights are their pairwise dissimilarities.  As output, graph embedding produces the coordinates of the data in a given dimensional space, which best preserves the relationships described in the graph.  It is typically performed in four steps: 1.) forming distance graphs from feature vectors, normalizing the graphs to account for distance scales individual to each dataset, assigning edges between the graphs, and running a graph embedding algorithm to produce the optimal coordinates in the new feature space.
		\\ \\
		Dimensionality reduction methods such as MDS, kernel PCA, manifold learning methods such as Laplacian Eigenmaps, Isomap, maximum variance unfolding and Sammon's mapping have all been applied as graph embedding procedures for space alignment.  However, not all methods are directly applicable since they usually assume a complete graph as input, which is not always available.  Additionally, there is no guarantee that the datasets have a similar intrinsic manifold structure.  Therefore, the primary purpose of this paper is neither dimensionality reduction nor intrinsic manifold discover, but finding a common space that reveals the relationships between different data sets by filling out the ``unknown" parts.
		\\ \\
		Another approach to space alignment is to seek linear transformations that map each space into a single space.  The advantage of linear methods is that out-of-sample  extension is straightforward. In other words, unseen data that are not involved in the alignment process can be directly mapped into the aligned space.  Additionally, linear methods generally give an idea about the feature-level relationships between different spaces from an analysis of the linear transformation coefficients.
		\\ \\
		A few approaches to space alignment (which were compared in this paper) are Constrained Laplacian Eigenmaps (which preserves locality), Orthogonal Procrustes Analysis (which gives an optimal linear transformation represented as an orthogonal matrix which best maps one data set to another that has been used widely in image registration) and PARAFAC2 which is similar to SVD for tensors.  One potential problem of Procustes Analysis is that eh dimensionality of the datasets must be equal.  \textbf{It should also be noted that the perfect alignment between the reference correspondence pairs does not always lead to good alignment for the rest of the data, which is analogous to overfitting in the context of classification.}
		\\ \\
		Two evaluation metrics were proposed in conjunction with the standard method of mean average precision (mAP).  One metric was proposed to measure the amount of deformation by determining the preservation of local neighborhoods through the transformation.  The second metric assesses alignment quality by determining how close correspondence pairs are matched in the new space.
		\\ \\
		
		\paperentry{Hertzberg2013SensorFusionStateRepManifolds}
		{Integrating generic sensor fusion algorithms with sound state representations through encapsulation of manifolds}
		{Manifold_Representation_Learning/Alignment/Hertzberg2013SensorFusionStateRepManifolds.pdf}
		{arg4}
		
		\paperentry{Cui2014GenUnsupManifoldAlignment}
		{Generalized Unsupervised Manifold Alignment}
		{Manifold_Representation_Learning/AlignmentCui2014GenUnsupManifoldAlignment/.pdf}
		{}\\
		The authors proposed a \textit{Generalized Unsupervised Manifold Alignment} (GUMA) method to build connections between differnt but correlated datasets without any known correspondences.
		\\ \\
		\textit{Manifold alignment} tries to build or strengthen the relationships of different datasets and ultimately project samples into a mutual embedding space, where the embedded features can be compared directly.  Since samples from different (even heterogeneous) datasets are usually located in different high dimensional spaces, direct alignment in the original spaces is very difficult.  On contrast, it is easier to align manifolds of lower intrinsic dimensions.
		\\ \\
		Supervised and semi-supervised methods require some known between-set counterparts as prerequisite for the transformation learning (aka labels or handcrafted correspondences).  In contrast, \textit{unsupervised manifold alignment} learns from manifold structures and avoids the need for hand-labeled correspondence.
		\\ \\
		In order to perform this matching, three assumptions are made: 1.) manifolds under the same theme (same action sequences of different persons usually imply a certain similarity in geometric structure), 2.) the embeddings of corresponding points from different manifolds should be as close as possible, and 3.) the geometric structures of both manifolds should be preserved in the mutual embedding space.  To obtain these goals, a three part objective function  is proposed.  Alternating optimization is performed between the correspondence matrix $\bm{F}$ and embedded points $\bm{P}_{x}$ and $\bm{P}_{z}$ to learn the correspondence points for matching and to corresponding embedding functions.
		\\ \\
		A few questions I have... Do the same actions imply similar manifold structure even for different modalities?  Also, how difficult would it be to extend this problem to multiple datasets?  What does this mean in terms of computational complexity? \textbf{Additionally, the authors make a note that there might be partial alignment cases, in which some points on one manifold might not correspond to any points on the other manifold (i.e. different resolutions) and that these points should be detected and not considered in the computation of the matching.  However, are these points still considered in the final embedding? i.e. are we losing information?}\\
		
		\paperentry{Tuia2015KernelManifoldAlignment}
		{Kernel Manifold Alignment}
		{Manifold_Representation_Learning/Alignment/Tuia2015KernelManifoldAlignment.pdf}
		{}\\
		The authors present a method for \textit{manifold alignment} (also denoted as \textit{feature representation transfer} or \textit{feature transformation learning}) 
		which addresses a few key deficiencies in manifold alignment approaches. Specifically, their \textit{kernel manifold alignment} (KEMA) (which can perform alignment or domain adaptation without corresponding pairs, only a few labeled examples in each of the domains) 1.) generalizes other manifold alignment methods, 2.) can align manifolds of different complexities by performing manifold unfolding along with matching, 3.) \textbf{can define a domain-specific metric to cope with multimodal specificities}, 4.) \textbf{can align data spaces of different dimensionalities}, 5.) is robust to strong nonlinear feature deformations and 6.) is closed-form invertible, which allows transfer across domains and data synthesis.
		\\ \\
		Roughly speaking, manifold alignment reduces to finding projections to a common latent space where all datasets show similar statistical characteristics.  There are three types of manifold alignment/ domain adaptation.  \textit{Unsupervised adaptation, semi-supervised adaptation, and supervised adaptation}.  Manifold alignment aims at concurrently matching the corresponding instances while preserving the topology of each input data domain, generally using a \textit{Graph Laplacian}.  While appealing, \textbf{these methods often require specifying a small amount of cross-domain sample correspondences.}
		\\ \\
		The method proposed in this paper is a kernel extension to Semi-supervised Manifold Alignment proposed by Wang et. al.  A key benefit to that method is that it can easily project data between domains by first mapping the data to the latent domain, then from there inverting back to the target domain.  Therefore, the method can be used for both domain adaptation and data synthesis.  Additionally, that method does not require data correspondence, only a few labeled samples from each domain.  Experiments show that the kernelized extension provided good performance, especially on data with strong nonlinear deformations. \\
		
				
		\paperentry{Liao2016ManAlignmentHSI}
		{A Manifold Alignment Approach for Hyperspectral Image Visualization With Natural Color}
		{Manifold_Representation_Learning/Alignment/Liao2016ManAlignmentHSI.pdf}
		{}\\
		In this paper, the authors apply manifold alignment to the transfer of HSI imagery to RGB for visualization.  The manifold embedding made bridged the gap between the high-dimensional spectral space of the HSI and the RGB space of the color imagery.  Once the embedding functions were learned, they were easily transferable to alternative images not included in the training set. 
		\\ \\
		Manifold alignment considers the mutual relationships of several data sets at the same time to generate a shared embedding space representing the common low-dimensional manifold shared among the datasets.  Of course, there is an assumption that seemingly disparate data sets are produced by similar generating processes and will share a similar underlying manifold structure.  Generally, there are two levels of manifold alignment.  \textit{Instance-level alignment} builds connections between instances from different datasets, but the alignment result is limited only to known instances and is difficult to generalize to new instances.  \textit{Feature-level alignment}, on the other hand, transforms features of different data sets to a common embedding space, which makes direct knowledge transfer possible.  The alignment result provides direct connections between features in different spaces, so it easily generalizes to new instances (although you have to assume your features in each space are accurate, discriminative representatives).
		\\ \\
		In this paper, feature-level manifold alignment was used to find direct mappings between HSI and RGB space.  First, a few matching pairs were denoted between the datasets (by hand).  Next, the manifolds were aligned in a shared embedding space (using \textit{Locality Preserving Projections}, a linear approximation of the nonlinear \textit{Laplacian eigenmaps}). Next, the inverse mapping could be used to transform the HSI imagery into the RGB colorspace. 
		\\ \\
		Experiments were conducted on HSI data over the Washington D.C. Mall and California Bay.  Results demonstrated good performance over four quantitative evaluation criteria.
		\\ \\
		\textbf{Manifold alignment based visualization only requires that a matching pair represents the same or similar class of objects/materials rather than being from the same geospatial location.  This is different from the traditional matching pair searching in image fusion/ image registration.  THIS SUGGESTS THAT EXACT IMAGE REGISTRATION IS NOT A NECESSARY CONDITION TO EXTRACT THE MATCHING PAIRS.}  \textbf{\textit{Connor note:} This signifies that weak labels could be used to discover the correspondence pairs.} \\
		
		\paperentry{Yang2016ManifoldAlignmentMultitemporalHSI}
		{Spectral and Spatial Proximity-Based Manifold Alignment for Multitemporal Hyperspectral Image Classification}
		{Manifold_Representation_Learning/Alignment/Yang2016ManifoldAlignmentMultitemporalHSI.pdf}
		{}\\
		The authors explore two methods of manifold alignment on HSI imagery.  The methods are compared to the SOA  method of LapSVM which is a semi-supervised, manifold-regularized support vector machine.
		\\ \\
		To fully exploit multi-temporal hyperspectral data, two issues must be addressed, 1.) high dimensionality and 2.) nonstationarity within temporal sequences.  To deal with the high dimensionality, \textbf{a dimensionality reduction process is required prior to analysis, with the goal of extracting the most relevant information to characterize data in a lower dimensional space (if the goal is classification, this suggests incorporating label information!!)}.  For classification, one of the benefitsof exploring the low-dimensional space is that fewer training samples are required to obtain a reliable classifier. 
		\\ \\
		The impact of nonstationarity phenomena is particularly significant due to the narrow spectral bands.   Spectral signatures are subject to change over time due to natural (e.g. seasonal phenology of vegetation or environment conditions) and disruptive (e.g. fire, anthropogenic) impacts.  Temporal nonstationarity can result in differences in the optimally reduces dimension space.  For this reason, additional ground reference information data are needed for classification.
		\\ \\
		Recently, nonlinear manifold learning was proposed as a successful dimensionality-reduction method to explore intrinsic information concealed in high-dimensional data and to capture nonlinearity Keith hyperspectral data.  \textbf{As data-driven methodologies, manifold learning methods often yield distance manifolds whose correspondence is difficult to model, particularly for environments that vary over space and time.}  One may attempt to develop a joint manifold representation by exploiting spectrally or spatially ``nearby" samples across two data sets.  However, spectral drift resulting from nonstationarity within a temporal sequence can lead to changes within a cluster of spectral data, and may not provide a reliable means for deriving joint data manifolds.  Assuming that the class types are the same and the intrinsic local structures of the classes contained in multiple images are similar, a proper cluster describing the same class may be obtained by aligning the local geometric structures of the spectral data.  Global spectral discrepancies associated with the class-dependent signature drift across a series of images may be reduced by grouping spectral neighbors within each image and mapping these clusters to an adequate common latent space supported by these local structures.
		\\ \\
		In this paper, the authors propose two manifold alignment (MA) techniques that involve aligning underlying local manifolds of temporally sequential data sets.  The first approach exploits a local-based manifold of a source image, considered to be the optial ``prior" manifold that cannot be intuitively reused by another image.  The second method links local manifolds of two images using bridging pairs (correspondence pairs) by considering spectral and spatial relationships between two temporal images.  The proposed methods extend graph-based semi-supervised learning and explore MA for the multitempoeral hyperspectral image classification task, while providing a domain adaptation framework for HSI analysis from the geometric learning point of view.  This paper explores the data geometry problem where only one labeled image is assumed to be available, and the goal is to classify another image.		
		\\ \\
		A data manifold is expected to be geometrically supported by data points.  The grate the number of data points, the better the data structure is represented, as long as the points are not redundant and are well distributed across the data manifold.  Popular manifold learning methods can generally be formulated in a graph embedding framework.  When exploring data manifolds that are developed separately, \textbf{it may be reasonable to assume that the intrinsic structures of classes contained in the sequential images are similar.} Manifold alignment has been investigated on a variety of tasks, however, \textbf{TO DATE, FEW STUDIES HAVE INVESTIGATED MA IN THE REMOTE SENSING COMMUNITY.}
		\\ \\
		The authors propose two methods for MA, one of which considers both spectral and spatial similarity by incorporating a spatial proximity term in the construction of the weight adjacency matrix.  Their methods are compared to the SOA LapSVM as well as several variations of KNN.  Their methods obviously outperforms the alternatives.
		\\ \\
		\textbf{A possible fruitful direction for extending these results would be to consider the use of submanifolds for classification tasks (this is great for my proposal!!!).  At a local scale, the regions with different desities can be viewed as submanifolds (within-class variatino in each cluster, similar to hierarchical methods.  This could be a good place for the hierarchical growing neural gas) which are combined at the global level. Such manifolds could be learned at the local scale and prove beneficial to our local MA concepts.  Local and global-based approaches concentrate on different characteristics of data manifolds.  Local approaches are more  favored in discriminating difficult classes, whereas global methods are advantageous for data representation over sequences of images.} \\
		
		
		\paperentry{Damianou2017ManifoldAlignmentDifferentDataView}
		{Manifold Alignment Determination: finding correspondences across different data views}
		{Manifold_Representation_Learning/Alignment/Damianou2017ManifoldAlignmentDifferentDataView.pdf}
		{}\\
		\textbf{Learning alignments is an important (and very challenging) task in multiview learning that we believe needs to be highlighted in order to stimulate further research.}  The author's present a methods for learning alignments between data points from multiple views of modalities, called \textit{Manifold Alignment Determination (MAD)}.  MAD requires only a few aligned examples  from which it is capable of recovering global alignment through probabilistic models.
		\\ \\
		Multiview learning is a strand of machine learning which aims at consolidating corresponding views into  a single model.  The underlying idea is to exploit correspondence between the views to learn a shared representation.  A very common scenario is when the views are coming from different modalities.  An example is when observing a conservation where both audio and video are separately being captures.  However, we can exploit the fact that the signals are aligned temporally ro obtain correspondence and thus to learn a joint representation for both signals.  This scenario, however, is based on the assumption that the data-points in each view are aligned, i.e. that for each video frame there is a single corresponding sound snippet. (Connor note: What if the sampling/ polling rates are different?? ) \textbf{In many scenarios we do not know the correspondence between the data-points and this uncertainty should be included in the multiview model. (Connor note: this would be the perfect scenario for cross-modality MIL to select the appropriate corresponding instances.)}
		\\ \\
		In this paper, the authors exploit the regularization of a probabilistic factorized mutliview latent variable model which allows the search to be formulated as a bipartite-matching problem.  Two different approaches are proposed: one myopic, which aligns data-poins in a sequential (iterative) fashion and a nonmyopic algorithm whcih produces the optimal alignment for a batch of data-points. Both implementations require a small number of initially aligned instances to act as a ``prior" to dictate what an alignment means in the particular scenario.  Experiments were conducted on simulated data and digits datasets.  The key insight from the paper is that the combination of a factorized latent space model and Bayesian learning naturally reflects alignments in multiview data. \\
		
		\paperentry{Shen2018ManifoldSensorFusionImageData}
		{Manifold learning algorithms for sensor fusion of image and radio-frequency data}
		{Manifold_Representation_Learning/Alignment/Shen2018ManifoldSensorFusionImageData.pdf}
		{arg4}
		
		\paperentry{Stanley2019ManAlignmentFeatureCorrespondence}
		{Manifold Alignment with Feature Correspondence}
		{Manifold_Representation_Learning/Alignment/Stanley2019ManAlignmentFeatureCorrespondence.pdf}
		{arg4}
		
		\paperentry{Hong2019LearnableManifoldAlignment}
		{Learnable manifold alignment (LeMA): A semi-supervised cross-modality learning framework for land cover and land use classification}
		{Manifold_Representation_Learning/Alignment/Hong2019LearnableManifoldAlignment.pdf}
		{}
	
	%%%%%%%%%%%%%%%%%%%%%%%%%%%%%% CHL %%%%%%%%%%%%%%%%%%%%%%%%%%%%%%%%%%%%%%%%%%%%%%%%%%%%%
	
	\subsection{Competitive Hebbian Learning}
	
	\paperentry
	{Rumelhart1985CHL}
	{Feature Discovery by Competitive Learning}
	{Manifold_Representation_Learning/CHL/Rumelhart1985CHL.pdf}
	{}
	
	\paperentry{Kohonen1990SOM}
	{The self-organizing map}
	{Manifold_Representation_Learning/CHL/Kohonen1990SOM.pdf}
	{}
	\newline
	The self-organizing map (SOM) creates spatially organized intrinsic representations of features.  It belongs to the category of neural networks which use ``competitive learning", or ``self-organization".  It is a sheet-like artificial neural network in which the cells become tuned to various input patterns through an unsupervised learning process.  Only a neighborhood of cells give an active response to the current input sample.  The spatial location or coordinates of cells in the network correspond to different modes of the input distribution. The self-organizing map is also a form of vector quantization (VQ).  The purpose of VQ is to approximate a continuous probability density function $p(\bm{x})$ of input vectors $\bm{x}$ using a finite number of codebook vectors, $\bm{m}_i$, $i=1,2,\dots,k$.  After the ``codebook" is chosen, the approximation of $\bm{x}$ involves finding the reference vector, $\bm{m}_c$ closest to $\bm{x}$.  The ``winning" codebook vector for sample $\bm{x}$ satisfies the following:
	\begin{align*}
		|| \bm{x} - \bm{m}_c|| &= \min_{i}|| \bm{x} - \bm{m}_{i} ||
	\end{align*}	
	\noindent
	
	The algorithm operates by first initializing a spatial lattice of codebook elements (also called ``units"), where each unit's representative is in $\bm{m}_i \in \mathbb{R}^{D}$ where $D$ is the dimensionality of the input samples $\bm{x}$.  The training process proceeds as follows.  A random sample is selected and presented to the network and each unit determines its activation by computing dissimilarity.	 The unit who's codebook vector provides the smallest dissimilarity is referred to as the \textit{winner}.
	
	\begin{align*}
		c(t) = \argmin_{i} d(\bm{x}(t),\bm{m}_{i}(t))
	\end{align*}
	\noindent
	
	Both the winning vector and all vectors within a neighborhood of the winner are updated toward the sample by 
	
	\begin{align*}
		\bm{m}_{i}(t+1) = \bm{m}_{i}(t) + \alpha(t) \cdot h_{ci}(t) \cdot [ \bm{x}(t) - \bm{m}_{i}(t) ] 
	\end{align*}
	\noindent
	where $\alpha(t)$ is a learning rate which decreases over time and $h_{ci}(t)$ is a neighborhood function which is typically unimodal and symmetric around the location of the winner which monotonically decreases with increasing distance from the winner.  A radial basis kernel is typically chosen for the neighborhood function as 
	
	\begin{align*}
		 h_{ci}(t) = \exp{\left( -\frac{||\bm{r}_{c} - \bm{r}_i ||^{2}}{2 \sigma^{2}(t)} \right)}
	\end{align*}
	\noindent
	where the top expression represents the Euclidean distance between units $c$ and $i$ with $\bm{r}_{i}$ representing the 2-D location of unit $i$ in the lattice.  The neighborhood kernel's bandwidth is typically initialized to a value which covers a majority of the input space and decreases over time such that solely the winner is adapted toward the end of the training procedure. \\
	
	\noindent
	The SOM essentially performs density estimation of high-dimensional data and represents it in a 2 or 3-D representation.  At test time, the dissimilarity between each unit in the map and an input sample are computed.  This dissimilarity can be used to effectively detect outliers, thus making the SOM a robust method which can provide confidence values for it's representation abilities. \\
	
	In this paper, the SOM was applied to speech recognition, but made note of previous uses in robotics, control of diffusion processes, optimization problems, adaptive telecommunications, image compression, sentence understanding, and radar classification of sea-ice. \\
	
 
	 \paperentry{Rauber2002GHSOM}
	 {The growing hierarchical self-organizing map: exploratory analysis of high-dimensional data}
	 {Manifold_Representation_Learning/CHL/Rauber2002GHSOM.pdf}
	 {The Growing Hierarchical Self-organizing Map (GHSOM) is an extension of the classical SOM.  It is an artificial neural network with a hierarchical architecture, composed of individually growing SOMs.  Layer 0 is composed of a single neuron representing the mean of the training data.  A global stopping criteria is developed as a fraction of the mean quantization error.  This means that all units must represent their respective subsets of data an a MQE smaller than a fraction of the 0 layer mean quantization error.  For all units not satisfying this criteria, more representation is required for that  area of the feature  space and additional units are added.  After a particular number of training iterations, the quantization errors are computed and the unit with the highest error is selected as the \textit{error unit}. The most dissimilar neighbor of the error unit is chosen is and a row/ column of nodes is injected between them.  The growth process continues until a second stopping criteria is met.  Any units still not satisfying the global criteria are deemed to need extra representation.  Child map are initialized below these units and trained with the subset of data mapped to its parent node.\\
	 	
 	\noindent
 	In conclusion, the GHSOM is a growing self-organizing map architecture which has the ability to grow itself until the feature space  is adequately represented.  For areas of the space needing a more specific level of granularity, a hierarchical structure is imposed to ``fill-in" areas of high density. \\
 	
 	\noindent
 	The GHSOM has been applied to the areas of finance, computer network traffic analysis, manufacturing and image analysis (Palomo 2017).
	}
 
	 \paperentry{Chiang1997HandWrittenWords}
	 {Hybrid fuzzy-neural systems in handwritten word recognition}
	 {Manifold_Representation_Learning/CHL/Chiang1997HandWrittenWords.pdf}
	 {}
	
	\paperentry{Frigui2009LandmineSOM}
	{Detection and Discrimination of Land Mines in Ground-Penetrating Radar Based on Edge Histogram Descriptors and a Possibilistic $K$-Nearest Neighbor Classifier}
	{Manifold_Representation_Learning/CHL/Frigui2009LandmineSOM.pdf}
	{}
	
	\paperentry{Fritzke1995GrowingNeuralGas}
	{A Growing Neural Gas Network Learns Topologies}
	{Manifold_Representation_Learning/CHL/Fritzke1995GrowingNeuralGas.pdf}
	{Abstract: An incremental network model is introduced which is able to learn the important topological relations in a given set of input vectors by means of a simple Hebb-like learning rule. In contrast to previous approaches like the "neural gas" method of Martinetz and Schulten (1991, 1994), this model has no parameters which change over time and is able to continue learning, adding units and connections, until a performance criterion has been met. Applications of the model include vector quantization, clustering, and interpolation. \\
		
	\noindent
	In contrast to SOMs and ``growing cell structures", which can project data onto non-linear subspaces which are chosen \textit{a priori}, the GNG is able to adapt its topology to match that of the input data distribution.  The growing process continues until a pre-defined level of quantization error has been reached. \\
	
	\noindent
	The base algorithm is outlined in Palomo (2017), \textit{Growing Hierarchical Neural Gas Self-Organizing Network}.		
	}
	
	\paperentry{Palomo2017GHNG}
	{The Growing Hierarchical Neural Gas Self-Organizing Neural Network}
	{Manifold_Representation_Learning/CHL/Palomo2017GHNG.pdf}
	{}
	\newline
	Abstract: The growing neural gas (GNG) self-organizing neural network stands as one of the most successful examples of unsupervised learning of a graph of processing units. Despite its success, little attention has been devoted to its extension to a hierarchical model, unlike other models such as the self-organizing	map, which has many hierarchical versions. Here, a hierarchical GNG is presented, which is designed to learn a tree of graphs.  Moreover, the original GNG algorithm is improved by a distinction between a growth phase where more units are added until no significant improvement in the quantization error is obtained, and a convergence phase where no unit creation is	allowed. This means that a principled mechanism is established	to control the growth of the structure. Experiments are reported, which demonstrate the self-organization and hierarchy learning abilities of our approach and its performance for vector quantization	applications.  Experiments were performed in structure learning, color quantization, and video sequence clustering. \\
		
	\noindent
	The aim of this method was to improve the adaptation ability of the Growing Hierarchical Self-Organizing Map proposed by Rauber (2002).  This was to be done through the extension of the Growing Neural Gas, which disposes of the fixed lattice topology enforced by the SOM.  Addtionally, the GNG learns a dynamic graph with variable numbers of neurons and connections.  The graph represents the input data in a more plastic and flexible way than the fixed-topology map.  	\\
	
	\noindent
	All clustering methods that learn a hierarchical structure have advantages even when used for non-hierarchical data. The learned hierarchical structure can be pruned at several levels, which yields alternative representations of the input data set at different levels of detail. This can be used to visualize a data set in coarser	or more detailed way. For vector quantization applications, the different pruning levels correspond to smaller or larger codebooks, so that a balance can be attained between the size of the codebook and the quantization error within the same hierarchical structure.\\
	
	\noindent
	The growing hierarchical neural gas (GHNG) model is defined as a tree of self-organizing graphs.  Each graph is made of a variable number of neurons or processing units, so that its size can grow or shrink during learning. In addition, each graph is the child of a unit in the upper level, except for the top	level (root) graph.  The training procedure is described by the following: \\
	
	\noindent
	Each graph begins with $H \geq 2$ units and one or more undirected connections between them.  Both the units and connections can be created and destroyed during the learning process. It is also not necessary that the graph is connected.  Let the training set be denoted as $\mathcal{S}$ with $\mathcal{S} \subset \mathbb{R}^{D}$, where $D$ is the dimensionality of the input space.  Each unit $i\in \{1, \dots, H \}$ has an associated prototype $\bm{w}_{i} \in \mathbb{R}^{D}$ and an error variable $e_i \in \mathbb{R}$, $e_i \geq 0$.  Each connection has an associated age, which is a nonnegative integer.  The set of connections will be notetd as $A \subseteq \{1, \dots,H \} \times \{1, \dots, H \}$.  The learning mechanism for the GHNG is based on the original GNG, but includes a novel procedure to control the growth of the graph.  First, a growth phase is performed where the graph is allowed to enlarge until a condition is met, which indicates that further growing would provide no significant improvement in the quantization error.  After that, a convergence phase is executed where no unit creation is allowed in order to carry out a fine tuning of the graph.  the leraning algorithm is provided in the following steps. \\
	
	\begin{enumerate}
		\item Start with two units ($H=2$) joined by a connection.  Each prototype is initialized to a sample drawn at random from $\mathcal{S}$.  The error variables are initialized to zero.  The age of the connection is initialized to zero.
		\item Draw a training sample $\bm{x}_{t} \in \mathbb{R}^{D}$ at random from $\mathcal{S}$.
		\item Find the nearest unit $q$ and second nearest unit $s$ in terms of Euclidean distance
		\begin{align*}	
		q &= \argmin_{i\in \{1,\dots,H \}} ||\bm{w}_{i}(t) - \bm{x}(t)  || \\
		s &= \argmin_{i\in \{1,\dots,H \} - \{q\} } ||\bm{w}_{i}(t) - \bm{x}(t)  ||
		\end{align*}
		\item Increment the age of all edges departing from $q$
		\item  Update the winning unit's error variable, $e_{q}$
		\begin{align*}
		e_{q}(t+1) = e_{q}(t) + || \bm{w}_q(t) - \bm{x}_{t} ||
		\end{align*}
	\end{enumerate}

	\noindent
	I believe the author's experimental approach did not take advantage of the method's strengths.  The author's only demonstrated experiments in vector quantization, and used corresponding metrics.  This method could be used to represent manifold  topology of differing dimensionality. This could be useful in HSI imagery, for example where different environment patches require manifold representations of various dimensionality.  Additionally, this could potentially be used to handle the sensor fusion problem with sensor loss/ drop-out. \\
	
	\paperentry{Sun2016GNGMotionDetection}
	{Online growing neural gas for anomaly detection in changing surveillance scenes}
	{Manifold_Representation_Learning/CHL/Sun2016GNGMotionDetection.pdf}
	{}
	
	\paperentry{LopezRubio2011GHPGraphs}
	{Growing Hierarchical Probabilistic Self-Organizing Graphs}
	{Manifold_Representation_Learning/CHL/LopezRubio2011GHPGraphs.pdf}
	{}
	
	
	\paperentry{Palomo2016GrowingNeuralForest}
	{Learning Topologies with the Growing Neural Forest}
	{Manifold_Representation_Learning/CHL/Palomo2016GrowingNeuralForest.pdf}
	{}
	
	\subsection{Deep Learning}
	
		\paperentry{Goodfellow2016DeepLearning}
		{Deep Learning}
		{Manifold_Representation_Learning/Autoencoders/Goodfellow2016DeepLearning.pdf}
		{}
		
		\paperentry{Haykin2009NeuralNetworks}
		{Neural networks and learning machines}
		{Manifold_Representation_Learning/Autoencoders/Haykin2009NeuralNetworks.pdf}
		{}
		
		\paperentry{Dai2017VariationalAutoencoder}
		{dden Talents of the Variational Autoencoder}
		{Manifold_Representation_Learning/Autoencoders/Dai2017VariationalAutoencoder.pdf}
		{}
		
		\paperentry{Rojas1996AssociativeNetworks}
		{Associative Networks}
		{Manifold_Representation_Learning/Autoencoders/Rojas1996AssociativeNetworks.pdf}
		{}

%%%%%%%%%%%%%%%%%%%%%%%%%%%%%% Manifold Dissimilarities %%%%%%%%%%%%%%%%%%%%%%%%%%%%%%%%%%%%%%
\section{Information Measures}

	\paperentry{Arandjelovic2005FaceRecManifoldDensityDivergence}
	{Face recognition with image sets using manifold density divergence}
	{Manifold_Representation_Learning/Information/Arandjelovic2005FaceRecManifoldDensityDivergence.pdf}
	{}
	
	\paperentry{Wang2008ManifoldManifoldDistance}
	{Manifold–Manifold Distance and its Application to Face Recognition With Image Sets}
	{Manifold_Representation_Learning/Information/Wang2008ManifoldManifoldDistance.pdf}
	{}
	

	
%%%%%%%%%%%%%%%%%%%%%%%%%%%%%% Manifold Regularization %%%%%%%%%%%%%%%%%%%%%%%%%%%%%%%%%%%%%%%
\section{Manifold Regularization}
	\paperentry{Tsang2007ManifoldRegularization}
	{Large-Scale Sparsified Manifold Regularization}
	{Manifold_Representation_Learning/ManifoldRegularization/Tsang2007ManifoldRegularization.pdf}
	{}
	
	\paperentry{Ren2017ManRegSAR}
	{Unsupervised Classification of Polarimetirc SAR Image Via Improved Manifold Regularized Low-Rank Representation With Multiple Features}
	{Manifold_Representation_Learning/ManifoldRegularization/Ren2017ManRegSAR.pdf}
	{}
	
	\paperentry{Belkin2006ManReg}
	{Manifold Regularization: A Geometric Framework for Learning from Labeled and Unlabeled Examples}
	{Manifold_Representation_Learning/ManifoldRegularization/Belkin2006ManReg.pdf}
	{}
	
	\paperentry{Ratle2010ManRegHSI}
	{Semisupervised Neural Networks for Efficient Hyperspectral Image Classification}
	{Manifold_Representation_Learning/ManifoldRegularization/Ratle2010ManRegHSI.pdf}
	{}
	
	\paperentry{Li2015ManRegReinforcementLearning}
	{Approximate Policy Iteration with Unsupervised Feature Learning based on Manifold Regularization}
	{Manifold_Representation_Learning/ManifoldRegularization/Li2015ManRegReinforcementLearning.pdf}
	{}
	
	\paperentry{Meng2018ManRegZeroShot}
	{Zero-Shot Learning via Low-Rank-Representation Based Manifold Regularization}
	{Manifold_Representation_Learning/ManifoldRegularization/Meng2018ManRegZeroShot.pdf}
	{}
	
%%%%%%%%%%%%%%%%%%%%%%%%%%% Multiple Instance Learning %%%%%%%%%%%%%%%%%%%%%%%%%%%%%%%%%%%%
\section{Multiple Instance Learning}

	\subsection{Multiple Instance Concept Learning}
	
		\paperentry{Bocinsky2019Thesis}
		{Learning Multiple Target Concepts from Uncertain, Ambiguous Data Using the Adaptive Cosine Estimator and Spectral Match Filter}
		{Multiple_Instance_Learning/Bocinsky2019Thesis.pdf}
		{}
		
		\paperentry{Jiao2017Thesis}
		{Target Concept Learning From Ambiguously Labeled Data}
		{Multiple_Instance_Learning/Jiao2017MIHE_Thesis.pdf}
		{}
		
		\paperentry{Mccurley2019SPIEWEMIComparison}
		{Comparison of hand-held WEMI target detection algorithms}
		{Multiple_Instance_Learning/Mccurley2019SPIEWEMIComparison.pdf}
		{}
		
		\paperentry{Bocinsky2019SPIEMIACEInitialization}
		{Investigation of initialization strategies for the Multiple Instance Adaptive Cosine Estimator}
		{Multiple_Instance_Learning/Bocinsky2019SPIEMIACEInitialization.pdf}
		{}
		
		\paperentry{Zare2015MILLandmineEMI}
		{Multiple instance dictionary learning for subsurface object detection using handheld EMI}
		{Multiple_Instance_Learning/Zare2015MILLandmineEMI.pdf}
		{}
		
		\paperentry{Cook2015Thesis}
		{Task driven extended functions of multiple instances (TD-eFUMI)}
		{Multiple_Instance_Learning/Cook2015Thesis.pdf}
		{}
		
		\paperentry{Cook2016LandmineTaskDriveneFUMI}
		{Buried object detection using handheld WEMI with task-driven extended functions of multiple instances}
		{Multiple_Instance_Learning/Cook2016LandmineTaskDriveneFUMI.pdf}
		{}
		
		\paperentry{Zare2016MIACE}
		{Multiple Instance Hyperspectral Target Characterization}
		{Multiple_Instance_Learning/Zare2016MIACE.pdf}
		{}
		
		\paperentry{Jiao2017MIHE}
		{Multiple instance hybrid estimator for learning target signatures}
		{Multiple_Instance_Learning/Jiao2017MIHE.pdf}
		{}
		
		\paperentry{Xiao2017SphereMIL}
		{A Sphere-Description-Based Approach for Multiple-Instance Learning}
		{Multiple_Instance_Learning/Xiao2017SphereMIL.pdf}
		{}
		
		\paperentry{Cheplygina2019MILSurvey}
		{Not-so-supervised: A survey of semi-supervised, multi-instance, and transfer learning in medical image analysis}
		{Multiple_Instance_Learning/Cheplygina2019MILSurvey.pdf}
		{}
		
		\paperentry{Li2017SmoothMIL}
		{Cross-validated smooth multi-instance learning}
		{Multiple_Instance_Learning/Li2017SmoothMIL.pdf}
		{}
		
		\paperentry{Cheplygina2016DissimilarityEnsemblesMIL}
		{Dissimilarity-Based Ensembles for Multiple Instance Learning}
		{Multiple_Instance_Learning/Cheplygina2016DissimilarityEnsemblesMIL.pdf}
		{}
		
		\paperentry{Wang2017DiversityMILActiveLearning}
		{Incorporating Diversity and Informativeness in Multiple-Instance Active Learning}
		{Multiple_Instance_Learning/Wang2017DiversityMILActiveLearning.pdf}
		{}
		
		\paperentry{Hajimirsadeghi2017MIClassificationMarkovNetworks}
		{Multi-Instance Classification by Max-Margin Training of Cardinality-Based Markov Networks}
		{Multiple_Instance_Learning/Hajimirsadeghi2017MIClassificationMarkovNetworks.pdf}
		{}
		
		\paperentry{Du2016MIChoquetIntegralFusion}
		{Multiple Instance Choquet integral for classifier fusion}
		{Multiple_Instance_Learning/Du2016MIChoquetIntegralFusion.pdf}
		{}
		
		\paperentry{Ilse2018AttentionBasedDeepMIL}
		{Attention-based Deep Multiple Instance Learning}
		{Multiple_Instance_Learning/Ilse2018AttentionBasedDeepMIL.pdf}
		{}
		
		
		\paperentry{Karem2016MILMultiplePositiveAndNegativeConcepts}
		{Multiple Instance Learning with multiple positive and negative target concepts}
		{Multiple_Instance_Learning/Karem2016MILMultiplePositiveAndNegativeConcepts.pdf}
		{}
		
		\paperentry{Xiao2017MIOrdinalRegression}
		{Multiple-Instance Ordinal Regression}
		{Multiple_Instance_Learning/Xiao2017MIOrdinalRegression.pdf}
		{}
		
		\paperentry{Gao2017CountGuidedWeaklySupervisedLocalization}
		{{C-WSL:} Count-guided Weakly Supervised Localization}
		{Multiple_Instance_Learning/Gao2017CountGuidedWeaklySupervisedLocalization.pdf}
		{}
		
		\paperentry{Li2017MultiviewMIL}
		{Multi-View Multi-Instance Learning Based on Joint Sparse Representation and Multi-View Dictionary Learning}
		{Multiple_Instance_Learning/Li2017MultiviewMIL.pdf}
		{}
		
		\paperentry{Cao2016VehicleDetectionMIL}
		{Weakly Supervised Vehicle Detection in Satellite Images via Multi-Instance Discriminative Learning}
		{Multiple_Instance_Learning/Cao2016VehicleDetectionMIL.pdf}
		{}
		
		\paperentry{Dietterich1996AxisParallelRectangles}
		{Solving the multiple instance problem with axis-parallel rectangles}
		{Multiple_Instance_Learning/Dietterich1996AxisParallelRectangles.pdf}
		{}
		
		\paperentry{Maron1998DiverseDensity}
		{A Framework for Multiple-instance Learning}
		{Multiple_Instance_Learning/Maron1998DiverseDensity.pdf}
		{}
		
		\paperentry{Maron1998MILSceneClassification}
		{Multiple-Instance Learning for Natural Scene Classification}
		{Multiple_Instance_Learning/Maron1998MILSceneClassification.pdf}
		{}
		
		\paperentry{Carbonneau2016MILSurvey}
		{Multiple Instance Learning: {A} Survey of Problem Characteristics and Applications}
		{Multiple_Instance_Learning/Carbonneau2016MILSurvey.pdf}
		{}
		
		\paperentry{Zhang2002EMDD}
		{EM-DD: An Improved Multiple-Instance Learning Technique}
		{Multiple_Instance_Learning/Zhang2002EMDD.pdf}
		{}
		
		\paperentry{Zare2015eFUMI}
		{Extended Functions of Multiple Instances for target characterization}
		{Multiple_Instance_Learning/Zare2015eFUMI.pdf}
		{}
		
		\paperentry{Jiao2018MIHE2}
		{Multiple instance hybrid estimator for hyperspectral target characterization and sub-pixel target detection}
		{Multiple_Instance_Learning/Jiao2018MIHE2.pdf}
		{}
		
	
	\subsection{Multiple Instance Classification}
	
		\paperentry{Cao2016VehicleDetectionMIL}
		{Weakly Supervised Vehicle Detection in Satellite Images via Multi-Instance Discriminative Learning}
		{Multiple_Instance_Learning/Cao2016VehicleDetectionMIL.pdf}
		{}
		
	
	\subsection{Multiple Instance Regression}
	
		\paperentry{Trabelsi2018FuzzyClusteringMILRegression}
		{Fuzzy and Possibilistic Clustering for Multiple Instance Linear Regression}
		{Multiple_Instance_Learning/Trabelsi2018FuzzyClusteringMILRegression.pdf}
		{}
		
		\paperentry{Ruiz2018MIDynamicOrdinalRegression}
		{Multi-Instance Dynamic Ordinal Random Fields for Weakly Supervised Facial Behavior Analysis}
		{Multiple_Instance_Learning/Ruiz2018MIDynamicOrdinalRegression.pdf}
		{}
		
%		\paperentry{}
%		{}
%		{Multiple_Instance_Learning/}
%		{}
%		
%		\paperentry{}
%		{}
%		{Multiple_Instance_Learning/}
%		{}
%		
%		\paperentry{}
%		{}
%		{Multiple_Instance_Learning/}
%		{}


	\subsection{Applications}
	
%		\paperentry{}
%		{}
%		{}
%		{}
		


%%%%%%%%%%%%%%%%%%%%%%%%%%%%%%%%%%%%%%% Fusion %%%%%%%%%%%%%%%%%%%%%%%%%%%%%%%%%%%%%%%%%%%%%
\section{Fusion}
	
	\subsection{Classical Approaches}
		\subsubsection{General Approach}
			\paperentry{Mohandes2018ClassifierCombinationTechniquesReview}
			{Classifiers Combination Techniques: A Comprehensive Review}
			{Fusion/Reviews/Mohandes2018ClassifierCombinationTechniquesReview.pdf}
			{} \newline
			\textbf{Should reference this paper for hierarchical representations of classifier combination methods.  Many good diagrams.  Combining expert opinions before making decisions can substantially increase the reliability of critical application systems such as medical diagnosis, security, and so on.}  Evidence from multiple classifiers can be combined on the data, feature, or decision level.  Classifier ensemble combination methods are known under many different names: multi-classifier combination, multi-classifier fusion, mixture of experts, and ensemble based classification, to name a few.  Since the most recent review, 8 years earlier, a few more methods for classifier combination were introduced, including: a signal strength based combination approach, a novel Bayes voting strategy, a modified weighted averaging technique using graph-theoretic clustering, a neural network based approach for training combination rules, weighted feature  combination, and hierarchical fuzzy stack generalization. \\
			\noindent
			Typical classifier combination algorithms begin with a set of scores from individual classifiers and produce a combined score for each class along with a final class label.  The problem then generalizes to finding a combination function with accepts a K dimensional score vector from each of the M classifiers, then produces a single, final classification score representing the selected class.  The M classifiers could be identical but use different feature sets as inputs, or use different parameter sets.  Alternatively, the classifiers could be different by nature but use the same set of input features.  The important distinction is that individual classifiers should not make identical erroneous decisions on the same observation set, i.e. they should provide complementary information. \\
			\noindent
			Most classifier combination techniques assume independence between features. \\
			\noindent Adaptive techniques for classifier fusion are mainly based on evolution or artificial intelligence algorithms.  They include neural network combination strategies and genetic algorithms as well as fuzzy set theory. Fusion using ANNs allows for non-linear combination of classifier outputs.  Adaptive methods also include adaptive weighting, associative switching, adaptive fuzzy integrals, mixture of local experts and hierarchical MLE.  Adaptive classifiers tend to do better than the non-adaptive type.  \\ 
			\noindent
			Describes the differences between bagging, boosting, AdaBoost, and HME. \textit{Bagging} is creating different datasets by bootstrapped versions of the original dataset (sampling with replacement).  In \textit{boosting}, individual classifiers are trained hierarchically to discriminate more complex regions of the feature space.  \textit{AdaBoost} is a variation of boosting which combines the outputs of weak classifiers into a weighted sum representing the final decision.  However, it is sensitive to noisy data and outliers.  Additionally, AdaBoost on the feature level falls victim to the curse of dimensionality.\\  
			\noindent
			Classifier fusion methods have been used on HSI data, to improve  accuracy when sensor data is subjected to drift, handwritten word recognition, sequential data with HMM classifiers only,etc. \\
			\noindent
			\textbf{The literature still lacks a comprehensive performance analysis of techniques for a given application.}  Important research questions still include: classifier post-processing before combination, using meta-heuristic algorithms to improve performance, such as using optimization algorithms with majority voting, showing the advantages/ disadvantages of using different strategies such as probabilistic, learning, decision based, or evidence based, additionally, finding the optimal number/ type of classifiers to fuse is an open question. \\
			
			
			\paperentry{Ruta2000OverviewClassifierFusionMethods}
			{An Overview of Classifier Fusion Methods}
			{Fusion/Reviews/Ruta2000OverviewClassifierFusionMethods.pdf}
			{}
			\newline
			``The objective of all decision support systems (DSS) is to create a model, which given a minimum amount of input data/information, is able to produce correct decisions."  ``the solution might be just to combine existing,  well performing methods, hoping that better results will be achieved.  Such fusion of information seems to be worth applying in terms of uncertainty reduction. Each of individual methods produces some errors, not mentioning that the input information	might be corrupted and incomplete. However, different methods performing on different data should	produce different errors, and assuming that all	individual methods perform well, combination of such multiple experts should reduce overall classification	error and as a consequence emphasize correct outputs."  ``Fusion of data/information can	be carried out on three levels of abstraction closely connected with the flow of the classification process: data level fusion, feature level fusion, and classifier fusion"  This paper focused on the later method of classifier fusion.  This process can essentially be categorized into two eruditions.  The first methods put emphasis on the classifier structure and do not do anything with the outputs until the combination process finds the best classifier or a selected group of classifiers.  Then their outputs are taken as a final decision or used for further processing.  The second category operates primarily on classifier outputs and can be further divided.  \newline  There are three possibles types of output labels generated by individual classifiers.  Crisp labels provide the lowest amount of information for fusion, as no information about potential alternatives is available.  Some additional information can be gleaned from labels in the form of class rankings.  However, fusion methods operating on classifiers with soft/fuzzy outputs can be expected to produce the greatest improvement in classification performance.  (Connor Note: This is valuable in terms of outlier rejection as well!).  The following explains an overview of classifier fusion methods operating on single class labels, class rankings, and fuzzy measures, respectively. \\ 
			\noindent
			\textbf{Methods operating on classifiers:}  \newline \textit{Dynamic Classifier Selection} (DCS) methods replect the tendency to extract  a single best classifier instead of mixing many different classifiers, by attempting to determing the single classifier which is most likely to produce the correct classification label for an input sample.  Only the output of the selected classifier is taken as a final decision.  The classifier selection process includes a partitioning of the input samples.  A classifier is is for each partition is selected locally.  All DCS methods rely on strong training data and by choosing only locally best classifier.  \textbf{They potentially lose some useful information from other well-performing classifiers.} Classifiers and their combination functions are typically organized in parallel and simultaneously and separately get their outputs as in input for a combination function.  A more reasonable approach, however, is \textbf{to organize all classifiers into groups and to apply different fusion methods for each group.}  A very important factor for the success of this method is the diversity of classifier types, training data, and methods involved.  \textbf{Any classification improvement may only be achieved if the total information uncertainty is reduced.}   This in turn depends on the diversity of information supporting different classification methods. \textbf{The same goal can be achieved by reduction of errors produced by individual classifiers.}  \textit{Hierarchical Mixture of Experts} (HME) is an example  of a fusion method whose strength comes from classifier's structure.  It is a supervised learning method based on the \textit{divide-and-conquer} principle.  It is organized as a tree-like structure of leaves.  Each leaf represents an individual expert in the network, each of which tries to solve a local supervised learning problem.  The outputs of the elements of the same node are partitioned and combined by the gating network and the total output of the node is given as a convex combination.  The expert networks are trained to increase the posterior probability according to Bayes rule.  A number  of learning algorithms can be applied to tune the mixture model.  \textit{Expectation-Maximization} (EM) is often used to learn the model parameters.  \textit{The HME technique does not seem to be applicable to large-dimensional datasets.} \newline \textbf{Fusing Single Class Labels:} Classifiers producing crisp, single-class labels (SCL) provide the least amount of useful information for the combination process.  The two most common techniques for fusing SCL classifiers are \textit{Generalized Voting} and \textit{Knowledge-Behavior Space} methods.  \\
			\noindent
			\textbf{Voting Methods:} \newline Voting strategies can be applied to a multiple classifier system assuming that each classifier gives a single class label as as output and no training data are available.  While there are many methods for combining these labels, they all lead to the following generalized voting definition.  Let the output of the classifiers form the decision vector $\bm{d} = [ \bm{d}_1, \bm{d}_2, \dots, \bm{d}_n ]^{T}$ where $\bm{d}_i \in \{ c_1, c_2, \dots, c_m, r \} $, $c_i$ denotes the class label of the i-th class and $r$ the rejection of assigning the input sample to any classes.  The binary characteristic function is defined as follows: (Have not input math)
			
%			\begin{align*}
%				B_{j}(c_i) = \begin{cases}
%				1 \text{ if } \bm{d}_j = c_i \\
%				0 \text{ if } \bm{d}_j \neq c_i 
%				\end{cases}
%			\end{align*}

			\noindent
			\textbf{Class Ranking Based Techniques:} \newline
			There are two primary methods for fusion of class rankings.  \textit{Class set reduction} (CSR) attempts to reduce the number of eligible classes by compromising between minimizing the class set size and maximizing the likelihood of inclusion in the true class.  This is typically performed through the \textit{intersection or union of neighborhoods}.  The second popular CSR method is \textit{Class Set Reordering} (CSRR) which tries to improve  the overall rank of the true class through techniques such as the \textit{Highest Rank Method}, \textit{Borda Count}, or \textit{Logistic Regression}.  
			
			\noindent
			\textbf{Soft-Label Classifier Fusion:} \newline
			Soft labels are outputs in the range $[0,1]$ and are typically referred to  as \textit{fuzzy measures}, which cover all known measures of  evidence: probability, possibility, necessity, belief, and plausibility.  Each of these measures are used to describe different dimensions of information uncertainty.  This class of fusion attempts to reduce the level of uncertainty by maximizing suitable measures of evidence.  Common methods for this type of fusion include: Bayesian, Fuzzy Integrals, Dempster-Shaffer Combination, Fuzzy Templates, Product of  Experts, and Artificial Neural Networks.  \textit{Bayesian} methods can be applied under the condition that the outputs of the classifier are expressed as posterior probabilities.  Typical methods of Bayesian fusion include Bayes Average and Bayes Belief Integration. \textit{Fuzzy Integrals} aim at searching for the maximal agreement between  the real possibilities  relating to objective evidence and the expectation, $g$, which defines the level of importance of a subset of sources.  The concept of fuzzy integrals arises from the $\lambda$-fuzzy measure, $g$, developed by Sugeno.  Common methods for Fuzzy Integration include the Sugeno Fuzzy Integral, Choquet Fuzzy Integral, and Webster Fuzzy Integral.  \textit{Product of  Experts} combines different probabilistic models of the same data by performing a weighted average of individual probability distributions. \\
			
			
			
			
			\paperentry{Tulyakov2008ReviewClassifierCombinationMethods}
			{Review of Classifier Combination Methods}
			{Fusion/Reviews/Tulyakov2008ReviewClassifierCombinationMethods.pdf}
			{}\newline
			This paper provides different categorizations of classifier combination methods, including ensemble and non-ensemble based techniques.  Simple combination functions include sum, wighted sum, max etc.  More complex functions include ANN, k-NN and the like.  Classifier fusion methods either assume a small number of classifier inputs/ sensors which obtain their performance benefit from information diversity, or they rely on large sets of ensembles which operate using different features/ areas of the input space, modalities, etc.  \\
			\noindent
			Fusion methods can operating on the \textit{feature level} aim to form a joint feature vector, and subsequently perform classification in the new feature space.  This approach can potentially provide additional information about the classes, but often requires a large training set due to the increased dimensionality of the joint feature space.  It should be noted that if the features used in the different classifiers are not related, then there is no real benefit for fusion on the feature level. Additionally, this method is not conceptually different from attempting to incorporate information from alternative sources into a single feature vector.\\
			\noindent
			Methods operating on the \textit{score level} fuse the outputs of individual classifiers.  It is possible, in this approach, that information is lost during combination.  This, however, is usually compensated by the lower computational cost of combination and superior training of the final system.  \\
			\noindent
			Classifiers can be organized based on their outputs.  The three main types are \textit{abstract, rank, and measurement}.  Abstract provides the lowest amount of information and is simply a single class label or unordered set of candidate classes.  Rank level outputs provide an ordered sequence of candidate classes, also called the \textit{n-best list}.  The candidate class at the first position is the most likely class while the class at the end is the most unlikely.  There are no confidence values attached to the class labels.  Measurement level outputs provide an ordered n-best  list along with corresponding confidence levels.  Measurement level outputs provide the most information of the three output types.  While fusion on the measurement level is desired, it might be difficult to achieve  since confidence values from different types of classifiers may not align easily (i.e. different ranges, scales, means, etc).\\
			\noindent
			Additionally, fusion methods can be classified based on their \textit{complexity}.  Simple combination rules such as sum, weighted sum, product, etc exhibit low complexity, while rank based methods such as the Borda count represent medium complexity.  An example of a high complexity combination type includes Behavior-knowledge spaces (BKS).\\
			\textbf{While there is substantial research on classifier ensembles, there are very few theoretical results explaining why they work.}  Most explanations use bias and variance.  However, such approaches can only provide asymptotic explanations of observed performance improvements.  \textbf{Ideally, the theoretical foundation for classifier ensembles should use statistical learning theory.}  The following few sections analyze performance improvements due to ensemble methods. \\
			\textbf{A good description of bagging a nd boosting is given.} \\
			Compared to ensemble-based classifiers, non-ensemble based methods attempt to combine heterogeneous classifiers which complement each other.  \textbf{The advantage of complementary classifiers is that each classifier can concentrate on its own small subproblem instead of trying to cope with the classification problem as a whole, which may be too difficult for a single classifier.  Ideally, the expertise of the specialized classifiers do not overlap.} \\
			Normalization methods are described and corresponding advantages/ dis-advantages are elucidated. \\
			A classic approach to classifier combination is the \textbf{\textit{Dempster-Shafer theory of evidence} (DS)}.  It was originally adopted by researchers in AI in order to process probabilities in expert systems, but has recently been adopted to sensor fusion and classifier combination.   DS theory is a generalization of the Bayesian theory of probability and differs in several aspects. First, DS theory introduces \textit{degrees of belief} that do not necessarily meet the mathematical properties of probabilities.  Second, it assigns probabilities to sets of possible outcomes rather than single events only.  Third, it considers probability intervals that contains the precise probability for sets of possible outcomes.  The two main ideas of DS theory are to obtain degrees of belief for one question from subjective probabilities for a related question, and Dempster's rule for combining such degrees of belief when they are based on independent items of evidence.  Dempster's rule of combination is a generalization of Baye's rule.  Dempter's rule defines the joint mass $m_{1,2}(X)$ for an outcome set $X$ as follows:
			
			\begin{align*}
				m_{1,2}(X) = 
				\begin{cases}
					0 \quad &\text{if} \quad X = \emptyset \\
					\frac{1}{1-K} \sum_{A_{i} \cap B_{j} = X}m_{1}(A_{i})m_{2}(B_{j}) \quad &\text{if} \quad X \neq \emptyset
				\end{cases}
			\end{align*}
			
			\noindent
			where
			
			\begin{align*}
				K = \sum_{A_{i} \cap B_{j} = X}m_{1}(A_{i})m_{2}(B_{j})
			\end{align*}
			
			\noindent
			DS theory has produced good results in document processing and is still used today.  \\
			Another complex, and popular, approach is the \textbf{\textit{Behavior-Knowledge Space (BKS)} method}.  BKS is a trainable combination scheme on the abstract level which requires neither measurements not ordered sets of candidate classes.  It tries to estimate the a posteriori probabilities by computing the frequency of each class for every possible set of classifier decisions, based on a given training set.  The result is a lookup table that associates the final classification result with each combination of classifier outputs, i.e. each combination of outputs in the lookup table is represented by its most often encountered class label.  Given a specific classifier decision $S_1, \dots, \S_M$ from $M$ individual classifiers, the a posteriori probability $\hat{P}(c_i|S_1, \dots, \S_M)$ of class $c_i$ is estimated as follows:
			
			\begin{align*}
				\hat{P}(c_i|S_1, \dots, S_M) = \frac{N(c_i|S_1, \dots, S_M)}{\sum_{j}  N(c_j|S_1, \dots, S_M)}
			\end{align*}
			\noindent
			where $N(c_i|S_1, \dots, S_M)$ counts the frequency of class $c_i$ for each possible combination of crisp classifier outputs. \\
			
			
			\paperentry{hackett1990multisensorfusion}
			{Multi-sensor fusion: a perspective}
			{Fusion/Reviews/hackett1990multisensorfusion.pdf}
			{}\newline
			\textbf{Multi-Sensor fusion deals with the combination of complementary and sometimes competing sensor data into a reliable estimate of the environment to achieve an output which is better than the modalities, individually.}  Multi-sensor fusion has been used  in target recognition, autonomous robot navigation, automatic manufacturing, scene segmentation, sensor modeling, and object recognition.  \textit{Sensor fusion combines the outputs from two or more devices that retrieve a particular property of the environment.}  Each sensor's measurements are, in general, imprecise and contain errors and uncertainties, so the consensus of multiple sensors measuring the same property can reduce uncertainty and reduce measurement ambiguity. Every sensor modality is sensitive to a different property of the environment; it is necessary to use multiple sensors in order to address these sensitivities. \textit{Sensor fusion deals with the selection of a proper model  for each sensor, and identification of an appropriate fusion method.}  There are several methods for combining multiple data sources.  A few are: deciding, guiding, averaging, Bayesian statistics, and integration.  Deciding is the use of a particular data source during a certain time of the fusion process, usually based on some confidence measure.  Averaging is the weighted combination of several data sources. This type of fusion ensures all sensors contribute to the fusion process, but not all to the same degree.  Guiding is the use of one or more sensors to focus the attention of another sensor on some part of the scene.  Integration is the  delegation of carious sensors to particular tasks, thus eliminating redundancy in sensor measurements.  The most simple method of fusion uses raw data of the same property obtained by multiple sensors of the same type.  Multi-sensor integration is the use of several sensors in a sequential manner. \\
			\noindent
			\textbf{Data from different sensors must be put into equivalent forms to allow for fusion.  In order for data from multiple sources to be fused, there must be some method to relate data points from one sensor with corresponding data points from the other sensors.}  The \textit{registered} data points allow for easy gathering of sensor information about one particular point in the scene.  \\
			\noindent
			Fusion methods can be broadly classified into two categories, \textit{direct} and \textit{indirect}.  Direct fusion combines raw sensor measurements while indirect methods transform the sensor data to be fused. \\
			\noindent
			\textbf{Before sensor measurements can be combined, we must ensure that the measurements represent the same physical entity. Therefore, we need to check the consistency of sensor measurements.}  One such method for checking measurement consistency is the \textit{Mahalanobis} distance.\\
			\noindent
			\textbf{Since each sensor is sensitive to a different modality, multiple sensors not only can provide multiple views of objects, but they can also impose more constraints to reduce the search space during matching.}\\
			
			\paperentry{zhang2010multisourceremotingsensingfusion}
			{Multi-source remote sensing data fusion: Status and trends}
			{Fusion/Reviews/zhang2010multisourceremotingsensingfusion.pdf}
			{}\\
			\textbf{Remote sensing data fusion, as one of the most commonly used techniques for fusion, aims to integrate the information acquired with different spatial and spectral resolutions from sensors mounted on satellites, aircraft and ground platforms to produce fused data that contains more detailed information than each of the sources, individually.}  \textbf{Fusing remotely sensed data, especially multi-source data, remains challenging due to reasons such as landscape complexity, temporal and spectral variations, and accurate data co-registration.} \textbf{\textit{Pixel level} fusion is the combination of raw data from multiple sources into single resolution data, which are expected to be more informative and synthetic than either of the input data or reveal the changes between data sets acquired at different times. \textit{Feature level} fusion extracts various features, e.g. edges, corners, lines, texture parameters, etc., from different data sources and then combines them into one or more feature maps that may be used instead of the original data for further processing. This is particularly important when the number of available spectral bands becomes so large that	it is impossible to analyze each band separately. Methods applied to extract features usually depend on the characteristics of the individual source data, and therefore may be different if the data sets used are heterogeneous. Typically, in image processing, such fusion requires a precise (pixel-level) registration of the available images. Feature maps thus obtained are then used as input to pre-processing for image segmentation or change detection. \textit{Decision level} fusion combines the results from multiple algorithms to yield a final fused decision. When the results from different algorithms are expressed as confidences	(or scores) rather than decisions, it is called soft fusion; otherwise, it is called hard fusion. Methods of decision fusion include voting methods, statistical methods and fuzzy logic based methods.} \textbf{PROVIDES A GREAT DESCRIPTION OF LiDAR USE DESCRIPTION FROM XIAOXIAO'S DISSERTATION.} \\
			\noindent 
			An undesirable property when applying pixel-level fusion techniques to the fusion of SAR and optical images is that either spectral features of the optical imageery or the microwave backscattering information is destroyed, or both simultaneously.  \\
			\noindent
			Applications: satellite Earth observations, computer vision, medical image processing, defense security, land use classification, Digital Surface Modeling (DSM), Digital Elevation Modeling (DEM), environmental monitoring, road mapping. archeology, building detection and reconstruction, etc. \\
			\noindent
			\textbf{For specific purposes, ancillary and terrestrial meta-data such as laser-scanners, GIS data, web-sensors, field survey data, economic consensus data, and meteorological data me be combined with remote sensing data to improve the performance of data fusion. }
		
	
		\subsubsection{Hierarchical Mixture of Experts}
		
			\paperentry{Jordan1993HME}
			{Hierarchical mixtures of experts and the EM algorithm}
			{Fusion/HME/Jordan1993HME.pdf}
			{}
		
			\paperentry{Yuksel2012TwentyYearsMixtureofExperts}
			{Twenty Years of Mixture of Experts}
			{Fusion/HME/Yuksel2012TwentyYearsMixtureofExperts.pdf}
			{}
			
			\paperentry{Beyer2009HeterogeneousMixtureOfExperts}
			{Heterogeneous mixture-of-experts for fusion of locally valid knowledge-based submodels}
			{Fusion/HME/Beyer2009HeterogeneousMixtureOfExperts.pdf}
			{}
			
			\paperentry{Shazeer2017SparselyGatedMixtureOfExperts}
			{Outrageously Large Neural Networks: The Sparsely-Gated Mixture-of-Experts Layer}
			{Fusion/HME/Shazeer2017SparselyGatedMixtureOfExperts.pdf}
			{}
		
		
		\subsubsection{Choquet Integral}
		
			\paperentry{Du2017Thesis}
			{Multiple Instance Choquet Integral For MultiResolution Sensor Fusion}
			{Fusion/Du2017Thesis}
			{}
		
			\paperentry{Smith2017ChoquetIntegralLandmine}
			{Aggregation of Choquet integrals in GPR and EMI for handheld platform-based explosive hazard detection}
			{Fusion/Choquet/Smith2017ChoquetIntegralLandmine.pdf}
			{}
			
			\paperentry{Smith2017GeneticProgrammingChoquetIntegral}
			{Genetic programming based Choquet integral for multi-source fusion}
			{Fusion/Choquet/Smith2017GeneticProgrammingChoquetIntegral.pdf}
			{}
		
			\paperentry{Du2019MIChoquetIntegral}
			{Multiple Instance Choquet Integral Classifier Fusion and Regression for Remote Sensing Applications}
			{Fusion/Choquet/Du2019MIChoquetIntegral.pdf}
			{}
		
			\paperentry{Anderson2017BinaryFuzzyMeasureChoquetIntegral}
			{Binary fuzzy measures and Choquet integration for multi-source fusion}
			{Fusion/Choquet/Anderson2017BinaryFuzzyMeasureChoquetIntegral.pdf}
			{}
			
			\paperentry{Du2018MultiResolutionSensorFusion}
			{Multi-Resolution Multi-Modal Sensor Fusion For Remote Sensing Data With Label Uncertainty}
			{Fusion/Choquet/Du2018MultiResolutionSensorFusion.pdf}
			{}
			
			\paperentry{Gader2004ChoquetIntegralLandmine}
			{Multi-sensor and algorithm fusion with the Choquet integral: applications to landmine detection}
			{Fusion/Choquet/Gader2004ChoquetIntegralLandmine.pdf}
			{}
			
			
			
		
		\subsubsection{Deep Learning}
		
			\paperentry{Jian2019AEInfraredVisibleFusion}
			{A Symmetric Encoder-Decoder with Residual Block for Infrared and Visible Image Fusion}
			{Fusion/DeepLearning/Jian2019AEInfraredVisibleFusion.pdf}
			{}
		
		\subsubsection{Graph-Based}
		
			\paperentry{Vivar2019MultiModalGraphFusion}
			{Multi-modal Graph Fusion for Inductive Disease Classification in Incomplete Datasets}
			{Fusion/GraphBased/Vivar2019MultiModalGraphFusion.pdf}
			{}
	%%%%%%%%%%%%%%%%%%%%%%%%%%%%%%% Subspace Fusion %%%%%%%%%%%%%%%%%%%%%%%%%%%%%%%%%%%%%%%%	
	\subsection{Subspace Learning}
	
		\paperentry{Hong2018CommonSubspaceLearningHSI}
		{CoSpace: Common Subspace Learning from Hyperspectral-Multispectral Correspondences}
		{Fusion/SubspaceLearning/Hong2018CommonSubspaceLearningHSI.pdf}
		{arg4}
		
		\paperentry{Zhang2014SemiSupManLearningFusion}
		{Semi-Supervised Manifold Learning Based Multigraph Fusion for High-Resolution Remote Sensing Image Classification}
		{Fusion/SubspaceLearning/Zhang2014SemiSupManLearningFusion.pdf}
		{arg4}
		
	%%%%%%%%%%%%%%%%%%%%%%%%%%%%%%% Fusion Metrics %%%%%%%%%%%%%%%%%%%%%%%%%%%%%%%%%%%%%%%%%%
	\subsection{Fusion Metrics}

	\subsection{Co-registration}
	
		\paperentry{Dawn2010SurveyRemoteSensingImageRegistration}
		{Remote Sensing Image Registration Techniques: A Survey}
		{Fusion/Dawn2010SurveyRemoteSensingImageRegistration.pdf}
		{}
	
		\paperentry{Brigot2016CoregistrationForestRemoteSensingImages}
		{Adaptation and Evaluation of an Optical Flow Method Applied to Coregistration of Forest Remote Sensing Images}
		{Fusion/Brigot2016CoregistrationForestRemoteSensingImages.pdf}
		{}
		
		\paperentry{Zitova2003SurveyImageRegistrationMethods}
		{Image registration methods: a survey}
		{Fusion/Reviews/Zitova2003SurveyImageRegistrationMethods.pdf}
		{}\\
		\textbf{Image registration is the process of overlaying images (two or more) of the same scene taken at different times, from different viewpoints, and/or by different sensors.}  Image registration can broadly be broken into two categories, \textit{area-based} and \textit{feature-based} and according to four basic steps of image registration: \textit{feature detection, feature matching, mapping function design, and image transformation and resampling.} \\
		\noindent
		Image acquisition can be divided into four methodologies.  \textit{Different viewpoints (multiview analysis)} involves collecting images of the same scene from different viewpoints.  \textit{Different times (multitemporal analysis)} collects images of the same scene acquired at different times, often on a regular basis.  The aim  is to find and  evaluate changes in the scene over time.  \textit{Differnt sensors (multimodal analysis)} collects images of the same scene through different sensors.  The aim is to integrate information obtained from different source streams to gain a more complex and detailed scene representation.  \textit{Scene to model registration} involves images of a scene and a corresponding model (such as digital elevation models (DEM)).  The objective is to localize the acquired image in the scene/ model and/ or compare them.\\
		\noindent
		\textit{Feature detection} involves finding salient and distinctive points of interest in an image (closed-boundary regions, edges, contours, corners, line intersections, etc.)  These features can be represented by their point representatives (centers of gravity, line endings, distinctive points) called \textit{control points} (CP).  Physically corresponding features can be dissimilar due to different imaging conditions and/or due to differing spectral sensitivities among sensors.  Features should be sufficiently robust and stable as to not be influenced by unexpected variations a nd noise. \textit{Feature  matching} involves pairing corresponding features  between the query and reference image.  Various features and (dis)similarity measures along with spatial relationships are employed during this process.  Classical area-based feature matching involves methods such as cross-correlation (CC) (which does not incorporate any structural analysis) and mutual information (MI) which measures the statistical dependency between two datasets.  Other feature-based matching methods include graph matching, clustering, Iterative Closest Points (ICP) and more.  Features should be invariant , unique, stable, and independent.  Area-based matching is preferable when images do not have many prominent details/ distinctive information.  However, they rely on the images having similar intensity functions.  Feature-based matching are typically applied when local structure information is more significant than the information carried by the image intensities.  These methods allow images of completely different natures to be registered (i.e. multi-modal).  However, the drawback of this class of matching is that the respective featues might be difficult to detect between images.   \textit{Transform model estimation} involves selecting the type and parameters of a \textit{mapping function} to align the sensed image with the reference. Mapping  models are either Global (which use all CPs) or Local (which decompose the image into patches and the function parameters are locally dependent to a patch.)  Global methods preserve shape (curvatures, angles, etc), while local methods allow for local image transformations, thus addressing local deformations.  Types of mapping functions include: bivariate polynomials, radial basis functions, elastic, fluid, diffusion based, level sets, and optical flow registration.  \textit{Image resampling and transformation} involves transforming an image by means of the mapping function.  The main limitation of image resampling is computational complexity, especially for high dimensional images. \\
		\noindent
		\textbf{Accuracy evaluation is a non-trivial problem, partially because errors can evolve into the registration process in each of its stages and partially because it is difficult to distinguish between registration inaccuracies and actual physical differences.}  Accuracy error is usually measured by local \textit{local error} (displacement of CP coordinates due to inaccurate detection), \textit{matching error} (measured by the number of false matches when establishing the correspondence between CP candidates), and \textit{alignment error} (the difference between the mapping model used for registration and the actual between-image geometric distortion).
		
		\noindent
		Applications: Image mosaicing, creating super-resolution images, integrating information into geographic information systems (GIS), in medicine, cartography, computer vision, classification, environmental monitoring \\
		
		\paperentry{Liang2014ImageRegistrationMutualInformation}
		{Automatic Registration of Multisensor Images Using an Integrated Spatial and Mutual Information (SMI) Metric}
		{Fusion/Liang2014ImageRegistrationMutualInformation.pdf}
		{arg4}
		
	
%		\subsubsection{Geocoding}
%	
%		\subsubsection{Similarity Measures}
%	
%		\subsubsection{Transformation, Interpolation, Re-sampling}
%	
%		\subsubsection{Conflation}
		
	\subsection{Multi-resolution Fusion}
	
	\subsection{Fusion of Mixed Data Types}
		
		\paperentry{Butenuth2007HeterogeneousGeospatialData}
		{Integration of heterogeneous geospatial data in a federated database}
		{Fusion/Butenuth2007HeterogeneousGeospatialData.pdf}
		{}
		
			
		\paperentry{Guo2019LVAforMultimodalLearningandSensorFusion}
		{Latent Variable Algorithms for Multimodal Learning and Sensor Fusion}
		{Fusion/Guo2019LVAforMultimodalLearningandSensorFusion.pdf}
		{}
		
		\paperentry{Zhang2019FusionHeteroEarthObsClimateZones}
		{Fusion of Heterogeneous Earth Observation Data for the Classification of Local Climate Zones}
		{Fusion/Zhang2019FusionHeteroEarthObsClimateZones.pdf}
		{}
	
	\subsection{Unsorted}
	
	\paperentry{Shen2016SpatioTemporalSpectralFusion}
	{An Integrated Framework for the Spatio–Temporal–Spectral Fusion of Remote Sensing Images}
	{Fusion/Shen2016SpatioTemporalSpectralFusion.pdf}
	{}
	
	
	\paperentry{}
	{}
	{}
	{}
	
%%%%%%%%%%%%%%%%%%%%%%%%%%%%%%%%%% Graph Processing %%%%%%%%%%%%%%%%%%%%%%%%%%%%%%%%%%%%%%%%%
\section{Data Processing on Graphs}	

\paperentry{Bronstein2017GeometricDeepLearning}
{Geometric Deep Learning: Going beyond Euclidean data}
{Manifold_Representation_Learning/NeuralNets/Bronstein2017GeometricDeepLearning.pdf}
{}

\paperentry{Nicolicioiu2019RecurrentSpaceTimeGraphNN}
{Recurrent Space-time Graph Neural Networks}
{Manifold_Representation_Learning/NeuralNets/Nicolicioiu2019RecurrentSpaceTimeGraphNN.pdf}
{}

\paperentry{Wu2019SurveyGraphConvolutionalNeuralNetworks}
{A Comprehensive Survey on Graph Neural Networks}
{Manifold_Representation_Learning/NeuralNets/Wu2019SurveyGraphConvolutionalNeuralNetworks.pdf}
{}

\paperentry{Zhou2019ReviewGraphNeuralNetworks}
{Graph Neural Networks: {A} Review of Methods and Applications}
{Manifold_Representation_Learning/NeuralNets/Zhou2019ReviewGraphNeuralNetworks.pdf}
{}

%%%%%%%%%%%%%%%%%%%%%%%%%%%%%%% Outlier Detection %%%%%%%%%%%%%%%%%%%%%%%%%%%%%%%%%%%%%%%%%%%
\section{Outlier/ Adversarial Detection}

%%%%%%%%%%%%%%%%%%%%%%%%%%%%%%%%%%%%%%% Army %%%%%%%%%%%%%%%%%%%%%%%%%%%%%%%%%%%%%%%%%%%%%%%%
\section{Army}

\paperentry{Hall2019ProbabilisticObjectDetection}
{Probabilistic Object Detection: Definition and Evaluation}
{Army/Hall2019ProbabilisticObjectDetection.pdf}
{A probabilistic object detection metric (PDQ - Probability-based Detection Quality) was proposed, thus defining the new task of defining probabilistic object detection metrics.  The ability of deep CNNs to quantify both \textit{epistemic} and \textit{aleatoric uncertainty} is paramount for deployment safety-critical applications.  PDQ aims to measure the accuracy of an image object detector in terms of its label uncertainty and spatial quality.  This is achieved through two steps.  First, a detector must reliably quantify its \textit{semantic uncertainty} by providing full probability distributions over known classes for each detection.  Next, the detectors must quantify spatial uncertainty by reporting \textit{probabilistic bounding boxes}, where the box corners are modeled as normally distributed.  A loss function was constructed to consider both label and spatial quality when providing a final detection measure.  The primary benefit of this method is that it provides a measure for the level of uncertainty in a detection. \\ \\ Is it possible to replace the probabilistic metric with a possibilistic one?  Could this be more effective at handling outlying cases?} 


\paperentry{Mahalanobis2019DSIACCharacterization}
{A comparison of target detection algorithms using DSIAC ATR algorithm development data set}
{Army/Mahalanobis2019DSIACCharacterization.pdf}
{The authors provided an initial characterization of detection performance on the DSIAC dataset using the \textit{Faster R-CNN} algorithm and \textit{Quadratic Correlation Filter (QCF)}.  Performance was evaluated on two datasets, ``easy'' and ``difficult'', where the difficulty was determined by number of pixels on target and local contrast.  Under difficult conditions, the Faster R-CNN algorithm achieved noteworthy performance, detecting as much as 80\% of the targets at a low false alarm rate of 0.01 FA/Square degree.  The dataset was limited by a lack of background diversity. }

\paperentry{Tanner2019DSIACNeuralNet}
{Fundamentals of Target Classification Using Deep Learning}
{Army/Tanner2019DSIACNeuralNet.pdf}
{A shallow CNN was utilized for ATR on the DSIAC MWIR dataset.  The goal of the study was to determine the range of optimal thresholds which would optimally separate the target and clutter class distributions defined by the CNN predictions (output of softmax), as well as determine an upper bound on the number of training images required for optimizing performance.  The shallow CNN (5 layers) and a Difference of Gaussians (DoG), which finds regions of high intensity on dark backgrounds were used to detect and classify targets.  The CNN could correctly classify 96\% of targets as targets and as few as 4\% of clutter as targets.  It was found that the DoG detector failed when the targets were small (long range) or if the overall image was bright (infrared taken during the daytime).  It was also determined that guessing the bright pixels were at the center of the targets was a bad assumption. (The brightest part of a target is not necessarily at its center.)}

\paperentry{Li2018CollaborativeSparsePriorsMultiViewATR}
{Collaborative sparse priors for multi-view ATR}
{Army/Li2018CollaborativeSparsePriorsMultiViewATR.pdf}
{}

\paperentry{Kokiopoulou2009GraphBasedClassificationMultipleObsSets}
{Graph-based classification of multiple observation sets}
{Army/Kokiopoulou2009GraphBasedClassificationMultipleObsSets.pdf}
{}


%%%%%%%%%%%%%%%%%%%%%%%%%%%%%%%% Segmentation %%%%%%%%%%%%%%%%%%%%%%%%%%%%%%%%%%%%%%%%%%%%%%
\section{Segmentation}
	\paperentry{Caselles1997GeodesicActiveContours}
	{Geodesic Active Contours}
	{Segmentation/Caselles1997GeodesicActiveContours.pdf}
	{}
	
	\paperentry{Alvarez2010MorphologicalSnakes}
	{Morphological Snakes}
	{Segmentation/Alvarez2010MorphologicalSnakes.pdf}
	{The authors introduce a morphological approach to curve evolution.  Snakes or curves iteratively solve partial differential equations (PDEs).  By doing so, the shape of the snake deforms to minimize the internal and external energies along its boundary.  The internal component keeps the curve smooth, while the external component attaches the curve to image structures such as edges, lines, etc.  Curve evolution is one of the most widely used image segmentation/ object tracking algorithms.  The main contribution of the paper is a new morphological approach to the solution of the PDE associated with snake model evolution.  They approach the solution using only inf-sup operators which has the main benefit of providing simpler level sets (0 outside the contours and 1 inside).}
	
	\paperentry{Marquez_Neila2014MorphologicalCurveBasedEvolution}
	{A Morphological Approach to Curvature-Based Evolution of Curves and Surfaces}
	{Segmentation/Marquez_Neila2014MorphologicalCurveBasedEvolution.pdf}
	{}

\newpage

\bibliography{references}
\bibliographystyle{plainnat}

\end{document}
