\chapter{Technical Approach}
In this chapter, we present the proposed method for discriminative dimensionality reduction using weakly-supervised learning. In the current state of the literature, few methods have been developed which address the problem of dimensionality reduction and manifold learning using weak labels under the multiple-instance learning framework.  Among the proposed methods, none address the possibility of nonlinearity in the underlying manifold representing bags or instances. To this end, this proposal presents a plan of research to fill this gap in the literature.  Specifically, we aim to provide a generalized framework for nonlinear dimensionality reduction under the multiple instance learning paradigm which promotes instance-level discriminability in the latent embedding space.  This proposal only considers the binary classification problem, which is motivated by target detection in remote  sensing. The approaches developed in this work will be directly compared to state-of-the-art approaches in the literature.  The remainder of this chapter is divided into two sections.  In the first section, we describe the initial work in investigating the efficacy of strictly supervised, nonlinear manifold learning methods for bag-level classification.  In the second section, we present the proposed approach for instance-level classification.

\section{Nonlinear Manifold Learning for Bag-Level Classification}

\subsection{S-LE Algorithm Description}

\subsection{SE-Isomap Algorithm Description}


\section{Nonlinear Manifold Learning for Instance-Level Classification}

\subsection{Diverse-Density/ Traditional Manifold Learning Based}

\subsection{Ranking-Loss Based}

The work by Wei et al. in \cite{Wei2016ImageBagGenerators} suggested that  for image classification tasks, certain formulations of MIL were better suited than others.  Algorithms such as miGraph, MIBoosting and miFV which assume non-i.i.d samples or take advantage of aggregating properties of bags tend to work better than those which adopt the standard assumption.  The authors of this work recommend miGraph with LBP bag generation or MIBoosting with Single Blob generation for image classification.  Additionally, classification performance tended to increase as the number of instances increased.

Choquet integral
Can we use to preference learning to discern instances which are likely to be positive and negative, then use these rankings for embedding?

Metric embedding is often realized through weakly-supervised learning, where instead of labels, the data is accompanied with sets of preferences. While this embedding could, conceptually be easy to implement at the bag level, enforcing instance embeddings is difficult due to the uncertainty on the labels.  However, one could (theoretically) use instance ranking as discussed in Section \ref{sec:MI_Ranking} to provide estimated preference sets on the instances for use in embedding.

\subsection{Dataset Table - Name, Summary, Justification/Use in proposed work}