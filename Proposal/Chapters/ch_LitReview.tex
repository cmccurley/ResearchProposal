\chapter{Background}

This chapter provides a literature review on Manifold Learning, including classic approaches, supervised and semi-supervised methods and uses of manifolds for functional regularization.  A review of the Multiple Instance Learning framework for learning from weak and ambiguous labels is provided. Additionally, this chapter reviews the existing literature in sensor fusion, focusing heavily on manifold alignment and fusion of heterogeneous and multi-resolution data types.  A brief review on data processing over graphs is also included.  Reviews describe basic terminology and definitions.  Foundational approaches are described and advances are addressed.

%%%%%%%%%%%%%%%%%%%%%%%%%%%%%%%% Manifold Learning %%%%%%%%%%%%%%%%%%%%%%%%%%%%%%%%%%%%%%%%%
\section{Manifold Learning}

\subsection{Classic Approaches}

%\subsubsection{Competitive Hebbian Learning}
%
%\subsubsection{Deep Learning}

\subsection{Supervised and Semi-Supervised Approaches}

\subsection{Manifold Regularization}


%%%%%%%%%%%%%%%%%%%%%%%%%%% Multiple Instance Learning %%%%%%%%%%%%%%%%%%%%%%%%%%%%%%%%%%%%
\section{Multiple Instance Learning}

\subsection{Multiple Instance Concept Learning}

\subsection{Multiple Instance Classification}

\subsection{Multiple Instance Regression}


%%%%%%%%%%%%%%%%%%%%%%%%%%%%%%%%% Sensor Fusion %%%%%%%%%%%%%%%%%%%%%%%%%%%%%%%%%%%%%%%%%%%
\section{Sensor Fusion}
	
	\subsection{Classic Approaches}
	
%		\subsubsection{Choquet Integral}
%		
%		\subsubsection{Hierarchical Mixture of Experts}
%		
%		\subsubsection{Deep Learning}
%		
%		\subsubsection{Graph-Based}
		
	\subsection{Feature-level Fusion}
	
	\subsection{Manifold Alignment}
		Manifold alignment is the process of matching two or more seemingly disparate datasets by mapping them to a joint latent space, while both preserving the qualities of each dataset and highlighting their similarities \cite{Wang2011ManifoldAlignment,Liao2016ManAlignmentHSI,Stanley2019ManAlignmentFeatureCorrespondence}.  
	
%	\subsection{Information Loss}
%	
%	\subsection{Geocoding}
%	
%	\subsection{Similarity Measures}
%	
%	\subsection{Transformation, Interpolation, Re-sampling}
%	
%	\subsection{Conflation}
	

%%%%%%%%%%%%%%%%%%%%%%%%%%%%%%%%% Graph Classification %%%%%%%%%%%%%%%%%%%%%%%%%%%%%%%%%%%%%%
\section{Data Processing on Graphs}

	\subsection{Classic Approaches}
	
	\subsection{Graph Convolutional Neural Networks}

%%%%%%%%%%%%%%%%%%%%%%%%%%%%%%%%%% Outlier Detection %%%%%%%%%%%%%%%%%%%%%%%%%%%%%%%%%%%%%%%%
%\section{Outlier/ Adversarial Detection}