%%-----------List of Symbols, Nomenclature or Abbreviation--------

%% Please note: a list of Symbols, terms, acronyms, etc. is not usually the best practice.
%% More often you should simply define an abbreviation the first time it is used.
%% If you DO need to include a list like this please notice that it must be paginated manually
%% by breaking it up into page size tables. Longtable will not wrap the definition properly if
%% it extends to a second line and a similar issue is encountered when the tabbing environment
%% is used. If you have a better way of meeting the Editorial Office requirements I'd love to hear about it.

\chapter*{LIST OF ABBREVIATIONS} \addcontentsline{toc}{chapter}{LIST OF ABBREVIATIONS} %Start
%writing here. This is optional.
\singlespacing
\begin{longtable}{l p{5in}} %if the terms in the first column are longer than 1.4 inches reduce the number 5 appropriately

AE & Autoencoder\\
APR & Axis-Parallel Rectangles\\
CA & Competency Aware\\
CCA & Canonical Component Analysis\\
CLFDA &  Citation Local Fisher Linear Discriminant Analysis\\
CRF & Conditional Random Field\\
CT & Computed Tomography\\
CHL & Competitive Hebbian Learning\\
DD & Diverse Density\\
DM & Diffusion Maps\\
DR & Dimensionality Reduction\\
DSIAC & Defense Systems Information Analysis Center\\
EM & Expectation-Maximization\\
EM-DD & Expectation-Maximization Diverse Density\\
FA & Factor Analysis\\
FN & False Negative\\
FP & False Positive\\ 
FPR & False Positive Rate\\
FPS & Frames per Second\\
FOV & Field of View\\
GAE & Graph Autoencoder\\
GLVM & General Latent Variable Model\\
GNG & Growing Neural Gas\\
GTM & Generative Topographic Mapping\\
GPS & Global Positioning System\\
HAC & Hierarchical Agglomerative Clustering\\
HS & Hyperspectral\\
HSI & Hyperspectral Image\\
ICA & Independent Component Analysis\\
IID & Independently and Identically Distributed\\
iPALM & Intertial Proximal Alternating Linearized Minimization\\
Isomap & Isometric Feature Mapping\\
KPCA & Kernel Principal Component Analysis\\
LDA & Fisher's Linear Discriminant Analysis\\
LE & Laplacian Eigenmaps\\
LiDAR & Light Detection and Ranging\\
LFW & Labeled Faces in the Wild Dataset\\
LLE & Locally Linear Embedding\\
LMNN & Large-Margin K-Nearest Neighbors\\
LFDA & Local Fisher Discriminant Analysis\\
LPP & Locality Preserving Projections\\
LVM & Latent Variable Model\\
LVQ & Learning Vector Quantization\\
MDS & Multi-dimensional Scaling\\
MI & Multiple Instance\\
MI-ALM & Multiple Instance Augmented Lagrangian Multiplier\\
MIDA & Multiple-Instance Discriminant Analysis\\
MidLABS & Multi-Instance Dimensionality reduction by Learning a mAximum Bag margin Subspace\\
MI-FEAR & Multiple-Instance Feature Ranking\\
MIL & Multiple Instance Learning\\
MIDR & Multiple Instance Dimensionality Reduction\\
MILES & Multiple-Instance Learning via Embedded Instance Selection\\
MLE & Maximum Likelihood Estimation\\
MLP & Multilayer Perceptron\\
MoFA & Mixture of Factor Analyzers\\
MoPPCA & Mixture of Probabilistic Principal Component Analysis\\
MWIR & Mid-wave Infrared\\
MVU & Maximum Variance Unfolding\\
NLPCA & Nonlinear Principal Component Analysis\\
PC & Principal Curve\\
PAC & Probably Approximately Correct\\ 
pAUC & Partial Area Under the Curve\\
PCA & Principal Component Analysis\\
PGM & Probabilistic Graphical Model\\
RBF & Radial Basis Function\\
RKHS & Reproducing Kernel Hilbert Space\\
ROC & Receiver-Operating Characteristic\\
ROI & Region of Interest\\
S-LE & Supervised Laplacian Eigenmaps\\
SOA & State-of-the-Art\\
SOM & Self-organizing Feature Map\\
SVD & Singular Value Decomposition\\
SVM & Support Vector Machine\\
TN & True Negative\\
TP & True Positive\\
TPR & True Positive Rate\\
t-SNE & t-Distributed Stochastic Neighbor Embedding\\
UMAP & Uniform Manifold Approximation and Projection\\
VAE & Variational Autoencoder\\
VQ & Vector Quantization\\

 \end{longtable}



\doublespacing

