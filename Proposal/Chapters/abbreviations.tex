%%-----------List of Symbols, Nomenclature or Abbreviation--------

%% Please note: a list of Symbols, terms, acronyms, etc. is not usually the best practice.
%% More often you should simply define an abbreviation the first time it is used.
%% If you DO need to include a list like this please notice that it must be paginated manually
%% by breaking it up into page size tables. Longtable will not wrap the definition properly if
%% it extends to a second line and a similar issue is encountered when the tabbing environment
%% is used. If you have a better way of meeting the Editorial Office requirements I'd love to hear about it.

\chapter*{LIST OF ABBREVIATIONS} \addcontentsline{toc}{chapter}{LIST OF ABBREVIATIONS} %Start
%writing here. This is optional.
\singlespacing
\begin{tabular}{l p{5in}} %if the terms in the first column are longer than 1.4 inches reduce the number 5 appropriately

BS & Bag-Space Paradigm\\
CLFDA &  Citation Local Fisher Linear Discriminant Analysis\\
CT & Computed Tomography\\
CHL & Competitive Hebbian Learning\\
DR & Dimensionality Reduction\\
DSIAC & Defense Systems Information Analysis Center\\
EM & Expectation-Maximization\\
ES & Embedded-Space Paradigm\\
FA & Factor Analysis\\
FP & False Positive\\ 
FPS & Frames per Second\\
GLVM & General Latent Variable Model\\
GTM & Generative Topographic Mapping\\
GPS & Global Positioning System\\
HAC & Hierarchical Agglomerative Clustering\\
HS & Hyperspectral\\
HSI & Hyperspectral Image\\
IS & Instance-Space Paradigm\\
Isomap & Isometric Feature Mapping\\
LDA & Fisher's Linear Discriminant Analysis\\
LE & Laplacian Eigenmaps\\
LiDAR & Light Detection and Ranging\\
LFW & Labeled Faces in the Wild Dataset\\
LLE & Locally Linear Embedding\\
LMNN & Large-Margin K-Nearest Neighbors\\
LPP & Locality Preserving Projections\\
MDS & Multi-dimensional Scaling\\
MI & Multiple Instance\\
MI-ALM & Multiple Instance Augmented Lagrangian Multiplier\\
MIC & Multiple Instance Classification\\
MIDA & Multiple-Instance Discriminant Analysis\\
MidLABS & Multi-Instance Dimensionality reduction by Learning a mAximum Bag margin Subspace\\
MIL & Multiple Instance Learning\\
MIDR & Multiple Instance Dimensionality Reduction\\
MWIR & Mid-wave Infrared\\
NCA & Neighborhood Component Analysis\\
PCA & Principal Component Analysis\\
RBF & Radial Basis Function\\
ROI & Region of Interest\\
S-LE & Supervised Laplacian Eigenmaps\\
SOA & State-of-the-Art\\
SOM & Self-organizing Feature Map\\
SVD & Singular Value Decomposition\\
TP & True Positive\\
t-SNE & t-Distributed Stochastic Neighbor Embedding\\
UMAP & Uniform Manifold Approximation and Projection\\

 \end{tabular}



\doublespacing

